\documentclass[greybox]{svmult}

\usepackage{proof}
\usepackage{amsthm}
\usepackage{amsmath}
\usepackage{amssymb}
\usepackage{cancel}
\usepackage{allrunes}
\usepackage{graphics}
\usepackage{verbatim}
\usepackage{color}
\usepackage{tikz}
\usepackage[curve]{xypic}
\usetikzlibrary{arrows}
\usepackage{algorithm2e}
%\usepackage{fontenc}
%
%\newtheorem{thm}{Theorem}[section]
%\newtheorem{cor}[thm]{Corollary}
%\newtheorem{lemma}[thm]{Lemma}
%\newtheorem{conjecture}[thm]{Conjecture}
%\newtheorem{definition}[thm]{Definition}
%\newtheorem{proposition}[thm]{Proposition}
%\newtheorem{example}[thm]{Example}
%\newtheorem{exer}[thm]{}

\graphicspath{{./white-japan/img/}}

\newcommand{\argref}{\Sigma}
\newcommand{\canon}{E_\phi^{M_\mu}}
\newcommand{\canonp}{E_\phi^{M'_\mu}}
\newcommand{\restr}[2]{{#1}|_{#2}}
\newcommand{\afs}[1]{\longrightarrow(#1)}
\newcommand{\three}{\{1,0,\hf\}}
\newcommand{\prl}{\vdash_{\L}}
\newcommand{\prb}{\vdash_{\cdia}}
\newcommand{\mm}{\mathcal M}
\newcommand{\agents}{\mathcal A}
\newcommand{\lang}{\mathcal L}
\newcommand{\langa}{\lang_1}
\newcommand{\langb}{\lang_2}
\newcommand{\langl}{\lang^{\tiny {\L}}}
\newcommand{\lblack}{\lang^{\cdia}}
\newcommand{\lwhite}{\lang^{\lozenge}}
\newcommand{\lctl}{\lang^{CTL}}
\newcommand{\black}{{\bf BLACK}}
\newcommand{\hf}{\frac 12}
\newcommand{\graph}{\Pi\times\Pi}
\newcommand{\graphs}{2^{(\graph)}}
\newcommand{\model}[1]{\models_{#1}}
\newcommand{\modluk}{\models_{\tiny\L}}
\newcommand{\ddia}[1]{\langle{#1}\rangle}
\newcommand{\dbox}[1]{[#1]}
\newcommand{\adia}{\lozenge}
\newcommand{\adiastar}{\lozenge^*}
\newcommand{\abox}{\square}
\newcommand{\cdia}{\blacklozenge}
\newcommand{\cbox}{\blacksquare}
\newcommand{\lad}{\langle\leftarrow\rangle}
\newcommand{\rad}{\langle\rightarrow\rangle}
\newcommand{\lab}[2]{\Pi_{#1}(#2)}
\newcommand{\ist}{{\bf L}}
\newcommand{\notf}{{\bf M}}
\newcommand{\und}{{\bf U}}
\newcommand{\ov}[1]{\overline {#1}}
\newcommand{\ovto}[1]{\overline {\ov #1}}
\newcommand{\acro}[1]{\textsc{#1}}

\newcommand{\anext}{\mathbf{A}\bigcirc}
\newcommand{\auntil}[1]{\mathbf{A}\,{#1}\,\mathcal{U}\,}
\newcommand{\euntil}[1]{\mathbf{E}\,{#1}\,\mathcal{U}\,}

%% OK. TAR DE MED SELV OM DE ER UN0DVENDIGE! 
\newcommand{\band}{\bigwedge}
\newcommand{\bor}{\bigvee}



\newcommand{\truls}[1]{\textcolor{magenta}{(truls: #1)}}
\newcommand{\sjur}[1]{\textcolor{cyan}{(sjur: #1)}}


\newcommand{\outa}[2]{#1^+(#2)}
\newcommand{\ina}[2]{#1^-(#2)}
\newcommand{\neu}[2]{#1^n(#2)}
\newcommand{\proto}{\mathbf C}
\newcommand{\clab}{\sf c}
\newcommand{\iterate}[1]{{\cal I}(#1)}
\newcommand{\cl}{cl}
\newcommand{\witness}[1]{{\sf w^{#1}}}
\newcommand{\buffer}[2]{{\sf b}_{#1}(#2)}
\newcommand{\sdepth}[2]{d_{#1}(#2)}
\newcommand{\shrink}[2]{\rho_{#1}(#2)}
\newcommand{\comp}[2]{C(#1,#2)}
\newcommand{\test}[1]{{#1}?}
\newcommand{\dlangm}{{\mathcal L}_{\acro{DDL}}}
\newcommand{\pis}[1]{{\mathbf e}_{#1}}
\newcommand{\carriers}[1]{Q_{#1}}
\newcommand{\kmod}[2]{{\cal K}_{(#1,#2)}}
\newcommand{\rels}[1]{{\sf R_{#1}}}
\newcommand{\update}[3]{{\mathcal U}_{#1}(#2,#3)}
\newcommand{\cons}[1]{{\textara{\ea}}(#1)}
\newcommand{\af}{{\sf F}}
\newcommand{\afn}{S}
\newcommand{\afe}{E}
\newcommand{\seq}[1]{\overrightarrow{#1}}
\newcommand{\basis}{basis }
\newcommand{\state}{state }
\newcommand{\views}{\mathcal B}
\newcommand{\viewsv}{\left(V_a\right)_{(a \in \agents)}}
\newcommand{\carrier}{Q_\views}
\newcommand{\sem}{\varepsilon}
\newcommand{\depth}[1]{|{#1}|^\adia}
\newcommand{\bisim}{\underline{\leftrightarrow}}

\title{Arguably argumentative: A formal approach to the argumentative theory of reason}


\begin{document}

\maketitle

\begin{abstract}

\end{abstract}

\section{Introduction}\label{sec:intro}

The idea that thought entails existence, or at least presupposes it, is a conceptual pillar of analytic philosophy. It is hard to disagree that is has a certain appeal: "I think, so I am, so I might as well go on thinking". For the sceptic, however, it also begs the following question: what is the "I" in such a line of thought? The armchair philosopher might be too busy with his thoughts to worry about it, but in the field of social psychology, particularly in the tradition going back to the work of George Herbert Mead and the Chicago school, it has well established pride of place (Mead 1967).

The best known theories developed in this field all agree that any "I" is essentially a social construction; you exist as a thinker only because you interact with other thinkers. In particular, social interaction is partly constitutive of self, not merely emergent from it. Consequently, reason itself is emergent from social contact, and the view that individual rationality is the basis for rational interaction must be rejected. Rather, interaction and reasoning should be seen as mutually dependent notions, on the basis of which more subtle notions of rationality can be explored.

This idea is both convincing and powerful, and it is becoming increasingly important to many different fields of research, including economy, law, biology and artificial intelligence (Blume and Durlauf 2001; Dworkin 1986; Waal and Ferrari 2010; Benthem 2011; Ossowski 2013). In all these research areas, there is a trend towards  viewing rationality as fundamentally embedded in a social context. Importantly, this context is seen as important not only because people are social and tend to interact, but also because \emph{who} they are, \emph{what} they want, and \emph{why} they want it, tends to depend on how they engage with each other and their environment. Hence the individual -- the \emph{agent} in the context of formal models -- is himself in need of more subtle analysis, in terms of the same structures that are used to describe important aspects of the economic, legal, environmental and computational world that contains him. 

To accommodate this point of view across different domains, we are in need of better theoretical foundations, allowing us to investigate the relationship between reasoning and interaction, taking into account that they are co-dependent and co-evolving. In paper, we address this challenge, and we do so using formal logic, drawing on tools and techniques developed in the context of multi-agent systems. The connection between various branches of social science and formal logic and computer science has received much attention in recent years, and it has led to a surge of interesting interdisciplinary research (Wooldridge 2009; Benthem 2008; Verbrugge 2009; Ditmarsch, van der Hoek, and Kooi 2007; Parikh 2001).

While much recent work in applied logic has been devoted to modelling agency and interaction, the standard starting point is still that agents reason in strict adherence to some common standard of correctness, specified by some given formal logic. Also, it is typically assumed that rational interaction emerges from the fact that agents are individually rational in some appropriate sense, for instance because they seek to maximize some given utility function. In this paper, we argue that in order to provide adequate formal foundations for rational interaction we need to depart from such reductionist assumptions. We point to the argumentative theory of reason, introduced in (Mercier and Sperber 2011), as an alternative approach, and we sketch a \emph{formal} representation of basic elements of this theory. We show, in particular, how existing tools and techniques in contemporary logic allows us to formulate systems of dynamic logic for multi-agent argumentation which can be used to encode and explore key theoretical aspects, as well as facilitate modelling of concrete systems.

The structure of the paper is as follows. In Section \ref{sec:arg} we present the necessary background on the argumentative theory of reason and the distinct notion of argumentation relied on in the theory of argumentation frameworks, as studied in artificial intelligence. We discuss the differences between these two notions of argumentation, and we argue that in order to use argumentation frameworks to arrive at logics for representing the argumentative theory, we need to conceptualize argumentation frameworks as subjective representations of semantic content, on the basis of which deliberation can take place. We argue that a fundamental question raised by the argumentative theory, which can then be analysed by formal logical tools, is the question of how argumentative deliberation works, and how it can sometimes create a common representation, a \emph{consensus} among participants. We propose, in particular, that the argumentative theory implicitly relies on, and suggests further study of, \emph{social} rationality constraint -- imposed at the deliberative level, and formulated with respect to the outcome of deliberation. Moreover, we argue that these constraints are \emph{not} reducible to corresponding notions of rationality that applies to individual reasoners, who are instead characterized by a distinct form of \emph{argumentative rationality}, in that they seek primarily to maximize their influence, to win as many arguments as possible.

In Section \ref{sec:ddl} we introduce \emph{deliberative Kripke frames}, a versatile formal semantics based on modal logic which gives us access to an abstract view of argumentative deliberation, well suited for further exploration of core theoretical aspects. We provide some examples of semantic modeling facilitated by this formalism, and we go on to present a simple modal language to reason about argumentative social processes.
We then motivate what we believe to be the main challenge for future work: how to characterize interesting notions of social rationality using theories in modal logic. We present a few preliminary suggestions in this regard, but argue that more work is needed to explore different theories, in languages of different complexity 
and expressive power. 

In Section \ref{sec:fut} we discuss the limitations of our own approach, and suggest directions for future work, and in Section \ref{sec:conc} we conclude. % We note that the formalism presented in this paper does not include representation of strategic aspects of deliberation -- it does not allow us to represent scenarios where agents form coalitions to coordinate their arguments in order to derive mutual benefit in terms of winning more arguments as a group.  To model this requires giving an account of higher-order deliberation, where coalitions first deliberate to decide how to choose a joint contribution to the higher-level deliberation scenario based on the differing views of their members. We think developing better techniques for expressing this enhancement is an important 

\section{Argumentative agents: Towards a semantics for individual reasoning based on argumentation}\label{sec:arg}

The argumentative theory of reason is formulated on the basis of a vast amount of experimental evidence, and its core idea is that the notion of argumentation can serve a foundational role in cognitive science. The theory holds that human reasoning evolved to facilitate efficient argumentation, and that the function of reason is not in arriving at logically correct forms of inference but to contribute to social interaction in such a way as to maximize the positive effect of  deliberation. This, for instance, can serve to shed light on why humans so often reason in a way that most theories would judge to be fallacious. According to the argumentative theory, they sometimes do so because fallacies can be useful in the context of argumentation.

It is important to note that the theory involves a notion of argumentation which is conceptually distinct from that found in traditional argumentation theory, going back to (Toulmin 2003) (first edition from 1958). In this research tradition, the theories developed to account for argumentation tend to be highly normative, focusing on recognizing and categorizing fallacies and on designing argumentation schemes and models that are meant to facilitate sound and rational reasoning, particularly regarding what arguments we should accept in a given scenario. The argumentative theory, on the other hand, asks us to look at human reasoning only as an element of more complex social processes that may or may not have outcomes that we judge desirable. In particular, to define more interesting normative forms of rationality, the argumentative theory suggests specifying them with respect to deliberative processes, not with respect to the reasoning processes taking place inside individuals.

Individual reasoning, on the other hand, is understood descriptively, in terms of how people reason, but in such a way that the theory explicitly tackles the normatively laden question of \emph{why} people reason the way they do. According to the argumentative theory, reason developed in order to facilitate successful argumentation, and the evidence we have about the nature of human reasoning supports the hypothesis that people reason in order to maximize their chances of exerting influence in the context of argumentative deliberation; They reason in order to win arguments.

This also explains why people often make ``mistakes" when they reason, and why they often make decisions that are not optimal, or even rational, in terms of a classical normative understanding. But as the argumentative theory explains, the outcome of deliberation can still resemble what traditional accounts of rationality deems to be desirable outcomes of individual reasoning. Hence it might be that established normative ideas about reason still have a role to play. They should be formulated differently, however, with respect to social processes. This latter point is not explicitly made in \cite{whyreason}, but the presentation given there is highly suggestive of it, as most of the examples and arguments used in favor of an argumentative view of reason relies on showing how the ``quality" of individual reasoning -- understood in a classical sense -- improves when the social conditions are favorable. We believe one of the most interesting questions raised by the argumentative theory is how to be more precise about the way in which constraints imposed at a social level can replace or at least support notions of individual rationality as a basis for exploring intelligent interaction.

In the following we will attempt to shed light on it by the use of multi-agent logic, and the first step is to identify the appropriate notion of agency. In particular, we need to provide a formalization of the \emph{argumentative agent}, the agent who reasons in order to win arguments. To do this, we will make use of argumentation frameworks, introduced in (Dung 1995). These are simple mathematical objects, essentially directed graphs, which facilitate the investigation of a whole range of interesting semantics (Baroni and Giacomin 2007). 

The theory of argumentation frameworks has been influential in the context of artificial intelligence (Rahwan 2009). It is capable of capturing many different semantic notions, including semantics for multi-valued and non-monotonic logics, logic programs and games (Dung 1995; Dyrkolbotn and Walicki 2013). More recently, the work in (Brewka, Dunne, and Woltran 2011) shows how argumentation frameworks can be used to provide a faithful (and computationally efficient) representation also of semantics that are formulated with respect to the much more fine-grained formalism of abstract dialectical frameworks (Brewka and Woltran 2010). For our project, it is also important to note that much recent work focuses on providing logical foundations the theory (Grossi 2010a; Grossi 2010b; M. W. A. Caminada and Gabbay 2009; Arieli and Caminada 2013; Dyrkolbotn and Walicki 2013; Dyrkolbotn 2013).

In our opinion, this makes argumentation frameworks highly suited as a technical starting point towards logics for argumentative deliberation. However, we propose to make use of them in a novel way, not to model actual argumentation scenarios, but to model agents' interpretations of semantic meaning -- in argumentative terms -- of the propositions that are up for debate.  In terms of each individual agent, using terminology from cognitive science, it places the argumentation framework at the informational level of cognitive processing, where previous work have already shown that logical tools can have a particularly crucial role to play, also serving to shed new light on established truths arrived at through empirical work, see e.g., (Stenning and van Lambalgen 2005). 

While much work on multi-agent argumentation has already been carried out in a formal and semi-formal context, we note that this work is mostly based on a traditional view of argumentation theory. For instance, we think this is implicit in recent formal work such as that of  (M. Caminada, Pigozzi, and Podlaszewski 2011; M. Caminada and Pigozzi 2011) and even more so in the survey of the field given in (Rahwan 2009). In our opinion, however, this view is inappropriate when attempting to formalize the argumentative theory.
The problem is that the representation of the argumentation scenario is fixed and not open to dispute and dynamic change, except with respect to the question of how it should be evaluated. But to model the argumentative theory, we need to depart from this starting point, since it is crucial that the basic representation of the surrounding semantic reality is itself a subjective construction, distinctly produced in each individual agent. This is why we use argumentation frameworks as models of the agents' internal view of the relevant arguments and how they are related.

In the following subsection we develop this idea in formal detail, starting with the necessary preliminaries regarding argumentation frameworks.

\subsection{Argumentation frameworks, agents and semantic views}\label{subsec:arg}

Given a set of atoms $\Pi$ -- acting as names of arguments -- an argumentation framework (AF) over $\Pi$ is a relation $E \subseteq \Pi \times \Pi$. Intuitively, an element $(x,y) \in E$ encodes the fact that arguments $x$ attacks argument $y$ and we can depict $E$ as a directed graph, giving a nice visualization of how the atoms in $\Pi$ are related as arguments, see Figure \ref{fig:af1} for an example. We introduce the notation $\outa E x = \{y \in \Pi \mid (x,y) \in E\}, \ina E x = \{y \in \Pi \mid (y,x) \in E\}$ and $\neu = \Pi \setminus \outa E x \cup \ina E x$.

Given an AF $E$, the purpose of an argumentation semantics is to identify, using the structure of $E$, the collection of sets of arguments that can be accepted if taken together, typically called \emph{extensions}, see e.g., \cite{....} For instance, if $E = \{(p,q),(r,p)\}$, then the semantics might prescribe $\{r,q\}$ as a set that can be accepted, since $r$ defends $q$ against the argument made by $p$ and $r$ is not in turn attacked. There are many different argumentation semantics, each catering to a different set of intuitions about what should be required for a given set of arguments to count as acceptable.

Given an AF $E$, it is natural to represent an extension $A \subseteq \Pi$ as a three-valued assignment $\clab_A:\Pi \to \three$ such that 
$$
\clab_A(x) = \begin{cases} 1 \text{ if } x \in A \\ 0 \text{ if } x \in \outa E A \\ \frac{1}{2} \text{ otherwise } \end{cases}
$$
Then a possible intuitive reading is that arguments in $A$ are regarded as true propositions, arguments attacked by one of these are regarded as false proposition, while all other arguments are taken to correspond to propositions which have an undecided semantic status. The three-valued representation of extensions also lead to an alternative view on argumentation semantics, due to \cite{camlab}, which takes a semantics to be a collection of three-valued assignments. In this way, a semantics for argumentation can be reasoned about using three-valued logic, an idea that has been explored in some recent work \cite{usSynthese,meESSLLI,Arieli}. This will be exploited in the coming sections, as we will rely on three-valued {\L}ukasiewicz logic when we reason statically, i.e., either about a given agents' subjective view or the current deliberative state.

In Figure \ref{fig:sem} we provide definitions of the most commonly known semantics based on argumentation frameworks. The logic introduced in the next section is parameterized by an argumentation semantics and the choice of such a semantics will not crucial be for our analysis in this paper. We note, however, that the \emph{admissible} semantics encode what seems to be minimal criteria of acceptability of arguments. Intuitively, it requires that an acceptable set must be free from internal conflict and that it must also be able to defend itself against all attacks. The other semantics in Figure \ref{fig:sem} are all based on the same idea, but adds other requirements that are less obviously reasonable. In the following we will assume only that whatever our argumentation semantics returns as an acceptable set, it is always also acceptable under the admissible semantics.

\begin{figure}
$\begin{array}{ll}
\text{\footnotesize{Admissible: }} & a(E) = \{\clab \in \proto(E) \mid  E^-(\clab^1) \subseteq \clab^0\} \\
\text{\footnotesize{Complete:}} & c(E) =  \{\clab \in \proto(E) \mid \clab^1 = \{x \in \Pi \mid E^-(x) \subseteq \clab^0\}\} \\
%\forall x \in \Pi: \\ & \clab(x) = 1 \iff \forall y \in E^-(x): \clab(y) = 0\} \\
\text{\footnotesize{Grounded:}} & g(E) = \{\bigcap c(E)\} \\
\text{\small{Preferred:}} \ \ & p(E) = \{\clab_1 \in a(E) \mid \forall \clab_2 \in a(E): \clab^1_1 \not \subset \clab^1_2\} \\
\text{\small{Semi-stable:}} \ \ & ss(E) = \{\clab_1 \in a(E) \mid \forall \clab_2 \in a(E): \clab^{\frac{1}{2}}_1 \not \supset \clab^{\frac{1}{2}}_2\} \\
\text{\small{Stable:}} \ \ & s(E) = \{\clab \in a(E) \mid \clab^{\frac{1}{2}} = \emptyset\} 
\end{array}$
\caption{Various semantics, defined for any $E \subseteq \Pi \times \Pi$}
\label{fig:sem}
\end{figure}

Towards the definition of an argumentative agent, let $\agents$ be a set of agent names. Then a \emph{view} for agent $a \in \agents$ is an AF $V_A \subseteq \Pi \times \Pi$. It encodes his interpretation of the semantic relationship between the arguments under consideration, specified in keeping with the idea that his reasoning is based on an \emph{argumentative} understanding of meaning. Then an \emph{argumentative state} is a tuple $(V_a)_{a \in \agents}$, associating a view with each agent. In this paper, we will assume for simplicity that the argumentative state remains the same throughout the course of deliberation, so that the views of the agents are not themselves subject to revision as the debate unfolds. This, however, can easily be extended by application of the dynamic framework developed in the next section.

To reason about AFs we will use a simple propositional language $\lang$, with negation and implication, as defined by the grammar below.
$$
\alpha := p \ \mid \ \neg \alpha \ \mid \alpha \to \alpha 
$$
where $p \in \Pi$. Then we can define static argumentative truth following {\L}ukasiewicz three-valued logic, by defining extensions of $\clab: \Pi \to \three$ inductively as follows
\begin{equation}\label{eq:lsem}
\begin{array}{l}
\overline \clab(p) = \clab(p) \text{ for } p \in \Phi \\
\overline \clab(\neg \alpha) = 1 - \overline \clab(\alpha) \\
\overline \clab(\alpha \to \beta) = min\{1,1-(\overline \clab(\alpha) - \overline \clab(\beta))\}
\end{array}
\end{equation}
Then, given an agent $a \in \agents$ with a view $V_A$, we can add the modality $\cdia_a$ to perform (boolean) meta-reasoning about the acceptance status of arguments on AFs, under some arbitrary semantics $\sem$. In particular, we get the following multi-agent language $\lblack$
$$
\phi := \cdia_a \alpha \ \mid \ \neg \phi \ \mid \ \phi \land \phi $$ where $\alpha \in \lang$ and $a \in \agents$. Given an argumentative state $\views = (V_a)_{a\in \agents}$, we define truth for formulas from $\lblack$ inductively as follows, for all formulas $\phi$
\begin{equation}\label{eq:asem}
\begin{array}{l}
\views \models_\sem \cdia_a \alpha \text{ if there is } \clab \in \sem(V_a) \text{ s.t. } \overline \clab(\alpha) = 1 \\
\views \models_\sem \neg \phi \text{ if not } \views \models_\sem \phi \\
\views \models_\sem \phi \land \psi \text{ if } \views \models_\sem \phi \text{ and } \views \models_\sem \psi 
\end{array}
\end{equation}

The crucial challenge that remains, and which will be addressed in the next section, is to find a mechanism for formally introducing an appropriate kind of multi-agent interaction and dynamics, suitable for representing the argumentative theory. This is the question we address in the following section.

\section{Argumentative deliberation: Towards formalization of rational interaction using dynamic argumentation logic}\label{sec:ddl}

Given a basis which encodes agents' views of the arguments, we are interested in the possible ways in which agents can deliberate, and how deliberation can serve to create new, socially defined, interpretations of the arguments, interpretations that are aggregated in a non-trivial way from the views of the individual agents.  We are interested, in particular, in characterizing and studying the \emph{effect} of deliberation on semantic meaning. This, in our opinion, is the crucial question that needs to be addressed in the search for new foundations for rational interaction.

The argumentative theory, for instance, makes the assertion that unsound reasoning on the individual level, motivated by agents' desire to win arguments, can lead to sound results when embedded in a suitable deliberative context. The challenge, then, is to attempt to characterize such suitable contexts, and to define conditions of deliberation that helps promote certain normative standards of soundness that can not, and should not, be imposed at the individual level. 

This is a challenge that points beyond the work already done in \cite{whyreason}, and we believe a possible criticism against the argumentative theory, as it stands today, is that it fails to make clear how the deliberative process can take unsound reasoning and produce a sound outcome.  What mechanisms are in play here, and what standards of soundness can we apply to them? Clearly, they cannot be reduced to mechanisms and standards that we should apply to reasoners individually, but need to be formulated in terms of the deliberative circumstances themselves.

This is clear from the reasoning used to support the argumentative theory, for instance when it is pointed out that groups of people tend to perform ``better" in reasoning tasks when each individual is challenged by people that have views which diverge from his own. That this is so has been established in much empirical work, but how are we to make sense of it as a normative claim, and how can we express it in a theoretical and formally precise setting? This is the question we embark on now, developing a dynamic logic framework for defining and studying various notions of deliberative soundness and rationality.

We propose to think of deliberation as a process which consists in moving between possible interpretations of the meaning of arguments, where the idea is that each move is caused by some deliberative event and each possible interpretation represents a possible aggregation of the views of the involved agents. In particular, given an argumentative state $\views$, we say that $q \subseteq \Pi \times \Pi$ is a \emph{deliberative state} for $\views$ if
\begin{equation}\label{eq:ds}
\bigcap_{a \in \agents}V_a \subseteq q \subseteq \bigcup_{a \in \agents}V_a
\end{equation}
We will use $\Pi(q) = \{x \in \Pi \mid \neu q x \not = \Pi\}$ to denote the set of arguments that appear in some attack from $q$. That is, $\Pi(q)$ contains the arguments that are not neutral with respect to all other arguments, according to the AF $q$. The constraint imposed by the definition of a deliberative state appears natural and hard to dispute. Indeed, it encodes the following principle about deliberation, which appears safe to assume in most, if not all, contexts:

\begin{quote}\label{principle}
If some information regarding the semantic relationship between two arguments is included in a deliberative state, the correctness of this information is endorsed by at least one agent. 
\end{quote}

In particular, we do not allow deliberation to result in interpretations that deviate from interpretations that are held unanimously by the agents. If everyone agrees on the meaning of an argument, the argument has this meaning, no matter how deliberation proceeds. As we will see, however, this does not entail that a possible  unanimity regarding the acceptance status of an argument is necessarily reflected in the view aggregated by deliberation. For instance, even if all agents agree that $p$ should be accepted, it is quite possible that $p$ will not be accepted after deliberation. If the agents differ in their account of \emph{why} $p$ should be accepted, in particular, deliberation might lead to the rejection of $p$, initial unanimity regarding $p$ notwithstanding, see Example ... for a concrete scenario. This in itself is interesting, and it suggests that subtle questions and phenomena arise when we attempt to be more precise about our normative claims regarding deliberative rationality.

Towards formal precision, we now define the core notion of a \emph{deliberative Kripke model}

\begin{definition}\label{def:dk}
Given a deliberative state $\views$, a deliberative Kripke model for $\views$ is a tuple $(Q,R)$ such that
\begin{itemize}
\item $Q$ is a set of deliberative states for $\views$
\item $R$ is a relation $R \subseteq Q \times Q$
\end{itemize}
\end{definition}

The idea is that the relation $R$ encodes a concrete deliberation based on the views in $\basis$. If $(q_1,q_2) \in Q$ the intuition is that there is some event that can take place in the deliberative state $q_1$ so that the aggregated views of the arguments is updated, taking us to the deliberative state $q_2$. In the first instance, we abstract away from events that can induce such a link, but this could be some agent presenting his point of view, or it could be some joint effort to reach a decision about some argument. The latter type of event will be encoded in Subsection \ref{sec:truls} where we take it as the basis for defining the class of open deliberations, which we propose as a possible candidate for a rationality constraint at the social level. For now, we are content with leaving the exact content of event unspecified. 

As an example of a deliberative model, consider the framework in Figure \ref{fig:del1}. Here, the argumentative state is problematic from the point of view of classical logic. In particular, we have $\views \models_\sem \neg \cdia_a x \land \neg \cdia_a \neg x$ under all $\sem$ from Figure \ref{fig:sem}, encoding that for agent $a$, the argument $x$ attacks itself and is not defeated. Hence it cannot be regarded as either true or false without leading to contradiction, and agent $a$ is prevented from reaching any classically sound conclusions about the status of either argument (since he also perceives $x$ to attack $y$). The agent $b$, on the other hand, has the view that $x$ and $y$ are in opposition to each other; If one of them is accepted the other must be rejected and vice versa, but he has no information which suggests choosing one over the other. In particular, we have $\views \models_\sem \cdia_b y \land \cdia_b \neg y$. Hence from his point of view, the semantic status of $x$ and $y$ remains unclear. Through deliberation, however, it is possible to arrive at a definite outcome which also resolves the inconsistency that $a$ believes to be present at $x$. One such scenario is depicted in Figure \ref{fig:sem}, where deliberation starts with the empty framework over $\{x,y\}$ and then proceeds by agent $a$ first putting forth his point of view, resulting in $q_1$, and then continuing with agent $b$ adding to this his own understanding, which results in the deliberative state $q_2 = V_a \cup V_b$. Here there is no problem, and the status of $x$ and $y$ has been definitely resolved, since $y$ must be accepted and then $x$ will be defeated, under all semantics from Figure \ref{fig:sem}, including classical logic, as encoded by the stable semantics.

\begin{figure}\label{fig:del1}
$\begin{array}{llllll}
V_a: \xymatrix{& x \ar[r] \ar@(lu,ld) & y}, & V_b: \xymatrix{x \ar@/_/[r] & y \ar@/_/[l] } \\ \\
q_0: \xymatrix{ x & y }, & q_1: \xymatrix{& x \ar@(lu,ld)  \ar[r] & y }, & q_2: \xymatrix{&  x \ar@(lu,ld)  \ar@/_/[r] & y \ar@/_/[l]}
\end{array}$
\caption{A deliberative model $(Q,R)$ over $\views = (V_a,V_b)$ with $R = \{(q_0,q_1),(q_1,q_2)\}$}
\end{figure}

This is an example of a scenario where everything runs smoothly and there is no controversy. In particular, both agents uncritically accept adding each others' points of view to the aggregated deliberative state, resulting in the union of their views emerging as the final outcome of deliberation. Things might not be so simple, however, and it is the more complicated scenarios that can benefit the most from logical modeling. It could be for instance, that agent $a$ has reservations about agent $b$'s interpretation of $y$ as an argument that also attacks $x$. If we are unsure about agent $a$'s stance in this regard, or, more generally, unsure about whether deliberation between $a$ and $b$ will eventually return a state where the $(y,x)$ edge is included, we can model this by introducing branching in the deliberative model. In particular, we could introduce a reflexive loop at $q_1$, to indicate the possibility that $b$'s perspective might come to be rejected. Then we have a branching deliberative model, and while it is still \emph{possible} to resolve the problems with the argumentative state, deliberation can also fail. 

To talk about deliberative models, allowing us to distinguish and identify situations such as these, we can use existing modal languages of varying expressive power. First we consider the following simple language $\lang_1$, which simply adds to $\lblack$ a modality for talking about one-step possibilities in deliberative models.

$$
\phi := \cdia \alpha \ \mid \ \cdia_a \alpha \ \mid \ \neg \phi \ \mid \ \phi \land \phi \ \mid \adia \phi
$$
where $\alpha \in \lang$. The definition of satisfaction for $\lang_1$ on deliberative models is then defined analogously to classical modal logic.

\begin{definition}\label{truth1}
Given an argumentation semantics $\sem$, an argumentative state $\views$ and a corresponding deliberative model $(Q,R)$, the truth of $\phi \in \lang_1$ on $(Q,R)$ at $q \in Q$ is defined inductively as follows
\begin{itemize}
\item $\views, (Q,R),q\models_\sem \cdia \alpha$ if $\exists \clab \in \sem(q): \overline \clab(\alpha) = 1$
\item $\views,(Q,R),q\models_\sem \cdia_a \alpha$ if $\exists \clab \in \sem(V_a): \overline \clab(\alpha) = 1$
\item ...
\item $\views,(Q,R),q \models_\sem \adia \phi$ if there is $q' \in Q$ s.t. $(q,q') \in R$ and $\views,(Q,R),q' \models_\sem \phi$
\end{itemize}
\end{definition}
 
We use the shorthand $\abox \phi := \neg \adia \neg \phi$ as usual. Let us consider the model from Figure \ref{fig:del1} as an example. Then it is easy to verify that $\views, (Q,R), q_0 \models_\sem \abox \abox \cdia \neg x$, expressing how two steps of deliberation will necessarily suffice to resolve $a$'s semantic problems with $x$ in this scenario. However, if we add a reflexive edge $(q_1,q_1)$ to this model, to encode uncertainty about whether agent $b$'s view will survive deliberation, we obtain only the weaker $\views, (Q,R),q_0 \models_\sem \adia \adia \cdia \neg x$.

This toy example illustrates that with the machinery now in place we can formally model a sense in which deliberation can sometimes turn individual views that classical logic deems problematic into deliberative states that are classically consistent. The requirement that deliberation should be organized in such a way that it \emph{always} function like this, can then at first sight appear as a good candidate for a normative notion of rationality imposed at the social level. It is a very strong notion, however, and it is problematic because it is not wholly social -- a requirement to the effect that the outcome of deliberation should always be classically consistent \emph{must} also involve restriction on what individual views we permit agents to endorse. This is easy to see intuitively. The case of a system with a single agent who believes something absurd, for instance, or a system with many agents where all agree on a contradiction, are obvious examples. The fact that deliberation alone cannot ensure consistency in such cases seems hard to dispute, and it also implies that classical consistency might not be suited as a rationality norm, not even at the social level. This, in particular, is true if we take seriously the argumentative theory, and attempt to create standards that can be applied to \emph{actual} reasoners, even if the circumstances are such that an outcome in complete adherence to classical principles cannot be achieved.

This is an insight that we can now formalize in terms of logic. In particular, we can formalize it as a \emph{social} rationality principle, a claim about deliberative models. There might be many different ways of doing this, but for now let us consider the following intuitive candidate axiom for classically rational deliberation:
\begin{equation}\label{crat}
\models_\sem \cdia \phi \lor \cdia \neg \phi
\end{equation}
It says that for all argumentative states, at all states in any corresponding deliberative model, it should be the case that every formula can either be accepted or rejected. This, we recall, is the conditions under which the underlying argumentation framework describes a classically consistent interpretation of the semantic atoms in the model. It is easy to see that this axiom is \emph{not} true on the class of all deliberative models. Hence it captures a non-trivial principle, a genuine restriction on deliberation. However, we also notice that for some argumentative states $\views$, there are \emph{no} corresponding deliberative models such that schema \ref{crat} holds. Hence if we impose it as an axiom of deliberation, we also restrict the class of permissible argumentative states, meaning that it is not a purely social, non-reductionist, approach to rationality. This, in particular, is the formal expression of the intuition that constraints on deliberation alone is not enough to ensure classical consistency in all circumstances. 

This recognition does not in itself suggest that schema \ref{crat} should be discarded. Rather, we think it will be interesting to investigate further in it what ways this and similar schemata can still permit \emph{more} variety in the reasoning patterns of individuals than what is allowed under normative theories that presuppose classical reasoning at the individual level. But we do believe that it suggests the need to also other notions, which can also provide normative guidance in cases when classical consistency is not possible to obtain. Moreover, we are now in a position to define two different kinds of social rationality principles, which will help us structure our future inquiries.

\begin{itemize}
\item Liberal principles: Rationality constraints that do not force us to restrict the set of argumentative states that we considered possible. 
\item Idealistic principles: Rationality constraints that require us to restrict the set of possible argumentative states.
\end{itemize}

Given a choice of logical language, we can also formalize these two kinds of principles, and we can explore them using logical tools. A liberal principle definable by $\lang_1$, for instance, is some collection $\Xi$ of formulas (possibly infinite), such that for all $\phi \in \Xi$, for all deliberative states $\views$, there is a model $(Q,R)$ for $\views$ such that for every state $q \in Q$ we have
$$
\views,(Q,R),q\models \phi
$$


  -- we have a theory which only allows us to consider certain well-behaved argumentative states

 that if we stipulate $\cdia p  \cdia \neg p$ as an axiom that should hold in every deliberative state, then 


 seems difficult to define the appropriate kind of deliberation, however, 

particularly strong normative claim, a that is seems unrealistic to enforce in all circumstances. Rather, we see it as an overriding aim, and we see the role of more refined normative notions to be that of maximizing the likelihood that this will happen. In particular, we will see that if we takes it as an absolute requirement on all forms of deliberation, then in some argumentative states, it is \emph{impossible} to deliberate successfully; a classically sound semantics cannot always be regained from deliberation, at least not as long as Principle (\ref{principle}) is observed.

The following is a formalization of the principle of classical soundness:

$$
\views \models \neg \cdia \neg \phi \to \cdia \phi
$$
If, in every deliberative state, the impossibility of $\neg \phi$ implies the existence of $\phi$, then the model maintains adherence to classical principles as a semantic invariant, both in terms of interpretation and evaluation.

The problem is that not all argumentative states give rise to any non-trivial deliberative models that satisfy this constraint. Hence the requirement is not possible to express merely as a requirement on the process of deliberation. It also requires imposing normative requirements on individual reasoners. For instance, in cases when all reasoners agree already, and they are all unsound in their reasoning, irrationality will prevail regardless of deliberation. Hence the strict classical assumption is not warranted, if we take seriously the stipulation that argument production, rather than soundness, is the function of reasoning (also in a normative sense).

However, we can consider other standards which are influenced by the classical standard, and perhaps seeks to reach it in as many cases as possible. For instance, we can stipulate that everyone should have their voices hears, so as to minimize the chance of one view being enforced against all others. Thus we arrive at the principle that deliberation should be organized in an egalitarian manner. How can we formalize such an idea?

One possible protocol is arrived at if we say that the majority opinion should always be adhered to, regarding what interpretation we should regard as successful following deliberation. This, however, appears to be rather too strict as well, albeit in a different way. In fact, if we break ties arbitrarily, it defines a unique deliberative state which is the eventual outcome of every debate: the one defined by taking the relationship between every $x,y$ to be determined by whether or not a majority of the agents believe that one attacks the other. In this case there is hardly any need for further logical analysis -- the outcome is given, once and for all. However, it is also a highly unsatisfactory account of deliberation, the purpose of which is precisely to allow positions and semantic interpretations to develop dynamically as agents interact in various ways, attempting to influence each other and the common ground. 

In fact, we believe it is not the role of logic to pinpoint the exact outcome of deliberation, but to explore rather the space of possibilities, describing restrictions by stipulating invariant formulas -- logical principles that should be true across all possibilities -- not by explicitly defining the set of possible outcomes by way of mathematical models.

There are several ways to do this, each reflecting a different level of ambition about what logical methods have to offer. First, we can use logic merely as a modeling language, allowing us to encode concrete scenarios in a precise way, to talk about them in formal language, and to employ model checkers to decide if given claims are true in a given scenario. In this regard, there is little doubt that logic has something to offer, and we show some examples of this in Subsection \ref{sec:modeling}. Second, we can use logic to study normative constraints that are mathematically defined and which restricts, or adds structure to, the models we allow ourselves to consider. This promises formalization of general theoretical constructs, not limited to modeling of concrete scenarios, that we can then study using formal tools. However, the extent to which this is a fruitful line of inquiry depends on the extent to which we can provide mathematically reasonable definitions of key concepts. In Subsection \ref{sec:truls} we argue that this line of research is also promising, and we do so by making a concrete suggestion as to how a normative notion of rational deliberation can be encoded by a special class of deliberative models that satisfies certain constraints. Thirdly, and most ambitiously, we can attempt to \emph{axiomatically define} notions of good argumentative deliberation, defining key concepts directly in the object language of some suitable logic, and studying them using reasoning systems developed for that logic. This raises both the question of what meaningful general notions we can capture by restrictions on models and also the question of how expressive logics we need to express these restrictions in the object language. It raises many interesting and tricky questions, for instance regarding the adequacy, decidability and computational complexity of the resulting systems. In Subsection \ref{sec:sjur} we argue that even this use of logic shows promise with regards to the argumentative theory, and we discuss possible axioms of deliberation that we believe deserve further scrutiny.

\subsection{Using deliberative models to represent instances of argumentative interaction}

\subsection{Using restricted classes of deliberative models to study notions of rationality and ``good" argumentative interaction}

\subsection{The search for axiomatic characterization of rational argumentative interaction}

 
There are two possible approached for the further study of rationality by way of deliberative logic, moving beyond the mere descriptive modeling of concrete deliberative scenarios, keeping track of the relevant information, to the normative questions of the \emph{design} of such processes.

\begin{itemize}
\item We can model and study normative constraints on deliberation semantically by considering restricted classes of deliberative models which are deemed to correspond to ``good" deliberation. We introduce one such restricted class in Subsection \ref{sec:truls} below, intended to model liberal, open deliberation, where it is explicitly forbidden to enforce given restrictions on the course of debate. 
\item We can define normative constraints corresponding to classes of models in the object language, using both known and new systems of temporal logic. We present an example of this in Subsection \ref{sec:sjur} below.
\end{itemize}



using logical languages that express properties of deliberative models? This is a question we must mostly leave for future technical work, but we make some preliminary observations. For instance, when it comes to the 

This, then, suggests that the notion is too strong. 


given a set of deliberative states $Q$ and a relation between them, encoding how deliberation unfolds, we can use modal logic to reason about the deliberative structure that results. This leads to the following definition of a deliberative Kripke model.



First we need to decide what we 
 It seems that the way in which deliberation takes place, and the mechanism by which a joint view is produced

 reach \emph{agreement} on how arguments are related. That is, we are interested in the set of all AFs that can plausibly be seen as resulting from a \emph{consensus} regarding the status of the arguments in $\Pi$. What restrictions is it reasonable to place on a consensus? It seems that while many restrictions might arise from pragmatic considerations, and be implemented by specific protocols for ``good'' deliberation in specific contexts, there are few restrictions that can be regarded as completely general. For instance, while there is often good reason to think that the position held by the majority will be part of a consensus, it is hardly possible to stipulate an axiomatic restriction on the notion of consensus amounting to the principle of majority rule. Indeed, sometimes deliberation takes place and leads to a single dissenting voice convincing all the others, and often, these deliberative processes are far more interesting than those that transpire along more conventional lines. However, it seems reasonable to assume that whenever \emph{all} agents agree on how an argument $p$ is related to an argument $q$, then this relationship is part of any consensus. This, indeed, is the only restriction we will place on the notion of a consensus; that when the AF $\af$ is a consensus for $\basis$, it must satisfy the following \emph{faithfulness} requirement.

\begin{itemize}
\item \emph{For all $p,q \in \Pi$, if there is no disagreement about $p$'s relationship to $q$ (attack/not attack), then this relationship is part of $\af$}
\end{itemize}

This leads to the following definition of the set $\cons \views$, which we will call the set of \emph{complete assents} for $\views$, collecting all AFs that are faithful to $\views$.

\begin{equation}\label{def:consensus}
\cons \views = \left\{\af \subseteq \Pi \times \Pi ~\left|~ \bigcap_{a \in \agents}V_a \subseteq \af \subseteq \bigcup_{a \in \agents}V_a\right.\right\}
\end{equation}

An element of $\cons \views$ represents a possible consensus among agents in $\agents$, but it is an \emph{idealization} of the notion of assent, since it disregards the fact that in practice, assent tends to be \emph{partial}, since it results from a dynamic process, emerging through \emph{deliberation}. Indeed, as long as the number of arguments is not bounded we can \emph{never} hope to arrive at complete assent via deliberation. We can, however, hope to initiate a process by which we clarify the status of more and more arguments, in the hope that this will approximate some complete assent. If we are lucky, it might even turn out to be \emph{robust}, in the sense that there is \emph{no} deliberative future where the results of current partial assent end up being undermined. Complete assent, however, arises only in the limit.

In practice, we can only every analyze this limit by looking at smaller parts of the whole, some partial consensus that has been obtained through deliberation. Hence we define a \emph{deliberative state} as a tuple $q = (q_S,q_E)$ such that $q_S \subseteq \Pi$ and 
\begin{equation}\label{eq:ds}
\bigcap_{a \in \agents}\restr {V_a} {q_S} \subseteq q_E \subseteq \bigcup_{a\in \agents} \restr{V_a}{q_S}
\end{equation}
So a deliberative state $q$ is an AF $q_E$ such that all attacks of $q_E$ are between arguments of $q_S$. Moreover, $q$ is faithfully generated from the views of the agents; $q_E$ only contains semantic information that is present in at least one agents' view. Now we are ready to define deliberative Kripke models.

\begin{definition}\label{main}
Given a basis $\views$, a deliberative Kripke model with respect to $\views$ is a tuple $(Q,R)$ where $Q$ is a set of deliberative states for $\views$ and $R \subseteq Q \times Q$ is a relation on $Q$.
\end{definition}
We will reason about deliberative Kripke models using the following language.
$$ \phi \quad ::= \cdia \alpha \ \mid \ \cdia_a \alpha ~|~ \neg \phi ~|~ \phi \wedge \phi ~|~ \ddia p \phi ~|~ \adia \phi$$ 
where $p \in \Pi$, $\alpha \in \lang$ and $a \in \agents$.

\section{Discussion and future work}\label{sec:fut}

\section{Conclusion}\label{sec:conc}

\end{document}



When and how deliberation might successfully lead to an approximation of complete assent is a question well suited to investigation with the help of dynamic logic. The dynamic element will be encoded using a notion of a deliberative event -- centered on an argument -- such that the set of ways in which to relate this arguments to arguments previously considered gives rise to a space of possible deliberative time-lines, each encoding the continued stepwise construction of a joint point of view. This, in turn, will be encoded as a monotonically growing AF $\af = (S,E)$ where $S \subseteq \Pi, E \subseteq S \times S$ and such that faithfulness is observed by all deliberative events. That is, an event consists in adding to $\af$ the agents' combined view of $p$ with respect to the set $S \cup \{p\}$. This leads to the following collection of possible events, given a basis $\views$, a partial consensus\footnote{These ``partial consensuses'' are sometimes referred to as ``contexts'' when they are used to describe graphs inductively, as we will do later.} $\af = (S,E)$ and an argument $p \in \Pi$:

\begin{equation}\label{eq:update}
\update \views \af p = \left\{X ~\left|~ \bigcap_{a \in \agents}\restr {V_a} {S \cup \{p\}} \subseteq X \subseteq \bigcup_{a \in \agents}\restr {V_a}{S \cup \{p\}}\right.\right\}
\end{equation}

To provide a semantics for a logical approach to deliberation based on such events, we will use Kripke models.

\begin{definition}[Deliberative Kripke model]\label{def:main} Given an argumentation semantics $\sem$ and a set of views $\views$, the deliberative Kripke models induced by $\views$ and $\sem$ is the triple $\kmod \views \sem = (\carriers \views, \rels \views,\pis \sem)$ such that
\begin{itemize}
\item $\carriers \views$, the set of points, is the set of all pairs of the form $q = (q_S,q_E)$ where $q_S \subseteq \Pi$ and $$\bigcap_{a \in \agents}\restr {V_a} {q_S} \subseteq q_E \subseteq \bigcup_{a \in \agents}\restr {V_a} {q_S}$$
The basis $\views$ together with our definition of an event, given in Equation \ref{eq:update}, induces the following function, mapping states to their possible deliberative successors, defined for all $p \in \Pi, q \in \carriers \views$ as follows
$$succ(p, q) \quad := \quad \{~(q_S \cup \{p\}, q_E \cup X) ~|~ X \in \update \views q p~\}$$ 
We also define a lifting, for all states $q \in \carriers \views$:
$$succ(q) \quad := \quad \{~q' \mid \exists p \in \Pi: q' \in succ(q,p)\}$$
\item $\rels \views: \Pi \cup \{\exists\} \to 2^{\carriers \views \times \carriers \views}$ is a map from symbols to relations on $\carriers \views$ such that 
\begin{itemize} \item $\rels \views(p) = \{(q,q') \mid q' \in succ(p,q)\}$ for all $p \in \Pi$ and
\item $\rels \views(\exists) = \{(q,q') \mid q' \in succ(q)\}$,
\end{itemize} \vspace{1em}
\item $\pis \sem: \carriers \views \to 2^{(3^\Pi)}$ maps states to labellings such that for all $q \in \carriers \views$ we have $\pis \sem(q) = (\pi_1,\pi_0,\pi_{\frac{1}{2}})$ with $$\pis \sem(q) = \{\pi \mid \pi_1 \in \sem(q), \pi_0 = \{p \in q_S \mid \exists q \in \pi_1: (q,p) \in q_E\}\}$$
\end{itemize}
\end{definition}

Notice that in the last point, we essentially map $q$ to the sets of extensions prescribed by $\sem$ when $q$ is viewed as an AF. We encode this extension as a three-valued labeling, however, following \cite{caminada06}. Notice that the default status, attributed to all arguments not in $q_S$, is $\frac{1}{2}$. The logical language we will use consists in two levels. For the lower level, used to talk about static argumentation, we follow \cite{Arieli,Sjur-SYNT} in using {\L}ukasiewicz three-valued logic. Then, for the next level, we use a dynamic modal language which allows us to express consequences of updating with a given argument, and also provides us with existential quantification over arguments, allowing us to express claims like ``there is an update such that $\phi$''. This leads to the language $\dlangm$ defined by the following \acro{bnf}'s

$$ \phi \quad ::= \cdia \alpha ~|~ \neg \phi ~|~ \phi \wedge \phi ~|~ \ddia p \phi ~|~ \adia \phi$$ 
where $p \in \Pi$ and $\alpha \in \lang$.

We also use standard abbreviations such that $\abox \phi = \neg \adia \neg \phi$, $\dbox p \phi = \neg \ddia p \neg \phi$ and $\cbox \alpha = \neg \cdia \neg \alpha$. We also consider that standard boolean connectives abbreviated as usual for connectives not occurring inside a $\cdia$-connective and abbreviations for connectives of {\L}ukasiewicz logic in the scope of $\cdia$-connectives.

Next we define truth of formulas on deliberative Kripke models. We begin by giving the valuation of complex formulas from $\lblack$, which is simply three-valued {\L}ukasiewicz logic.

\begin{definition}[$\alpha$-satisfaction] For any three-partitioning $\pi = (\pi_1,\pi_0,\pi_{\frac{1}{2}})$ of $\Pi$, we define 
\begin{align*}
\overline \pi(p) &= x \text{ s.t } p \in \pi_x \\
\overline\pi(\neg\alpha) &= 1 - \overline\pi(\alpha)\\
\overline\pi(\alpha_1 \to \alpha_2) &= \min \{1, 1 - (\overline\pi(\alpha_1) - \overline\pi(\alpha_2))\} 
\end{align*}
\end{definition}

Now we can give a semantic interpretation of the full language as follows.

\begin{definition}[$\dlangm$-satisfaction]\label{def:ddlm}
Given an argumentation semantics $\sem$ and a basis $\views$, truth on $\kmod \views \sem$ is defined inductively as follows, in all points $q \in \carriers \views$.
\begin{align*}
\kmod \views \sem, q \vDash \cdia \alpha & \iff \quad \text{there is } \pi \in \pis \sem(q) \text{ s.t. } \overline\pi(\phi) = 1 \\
\kmod \views \sem, q \vDash \neg \phi \quad & \iff \quad \text{not } \kmod \views \sem, q \vDash \phi \\
\kmod \views \sem, q \vDash \phi \wedge \psi & \iff \quad  \text{both } \kmod \views \sem, q \vDash \phi \text{ and } \kmod \views \sem, q \vDash \psi \\
\kmod \views \sem, q \vDash \ddia \pi p & \iff \quad \exists (q,q') \in \rels \views(p): \kmod \views \sem,q' \vDash \phi \\
\kmod \views \sem, q \vDash \adia \phi & \iff \quad \exists (q,q') \in \rels \views(\exists): \kmod \views \sem,q' \vDash \phi \\
\end{align*}
\end{definition}

sjur{Needs rewriting, but this is easy.}


I have already contributed to preliminary work on developing logics for argumentation based on such a perspective, and I will aim develop to develop this further (Dyrkolbotn and Pedersen forthcoming). Importantly, I do not think the role of logic is to pinpoint exactly the desired outcome of various deliberative processes, but to provide tools for analysing various restrictions and core principles. For instance, the formalism should allow us to explore various social rationality constraints, using languages that enable us to talk about the branching structure of possible deliberative futures.
The research goals and the timetable
I will develop modal logics that have a branching-time temporal nature, where deliberation on the basis of agents' different views generate temporal Kripke structures where states contains argumentation frameworks encoding the current consensus. This will enable me to study argumentative deliberation by investigating the expressive power and meta-logical properties of various formalisms which include modalities for talking about the semantics status of arguments at a given state, various kinds of deliberative actions that agents might perform to influence the future course of deliberation, and temporal and/or fixed point operators which can provide access to a birds eye view of the temporal structure itself.
With respect to the resulting logical systems, I will address the typical formal questions, such as complexity of model checking, decidability and complexity of the validity problem, axiomatization and expressive power. I will keep an eye out for connections with related formal work, and I will be particularly interested in applying my results to other  formalisms. In particular, I think the argumentative theory can provide important cues as to how one should model important aspects of the emergence of agents' identity, including an account of how coalitions develop, and of how agents come to form and revise their goals and of how they develop strategies for attempting to reach them. 
The basic idea that differences between agents – their individuality – is dynamically emergent and dependent on the system itself, underlies most if not all of the work I have carried out with regards to multi-agent logic. By proposing a project that focuses on formal modelling of the foundational question of why agents reason the way they do, I hope to arrive at a better template for developing models that can incorporate core insights about the nature of rational interaction and agency that have so far been largely overlooked, particularly in formal models.

\section{Rationality and reason revisitied -- the need for new foundations}


\sjur{Why all our references from various fields are implicitly in need of some common understanding of rational interaction that is not reducible to individual rationality.} 

\section{The argumentative theory of reason}

\sjur{What Mercier and Sperber's theory is and what it is not. It introduces the idea that agents are utility-maximizing in such a way that they seek to win arguments. This leads to behaviour typically seen as irrational and/or illogical. However, this irrationality might still lead to outcomes of interaction that appear reasonable/rational, sometimes also precisely *because* individual agents are not themselves rational in the classical sense. Hence the theory contains the idea that social rationality is not reducible to individual utility-maximizing (this is trivially true; it does not make sense to talk about a society that tries to "win" arguments, so the standard is obviously different at the social level). However, a weakness of Mercier and Sperber's theory is that they remain vague about what exactly the social rationality consists in. They only appeal to intuitions that appear classical, but do not inform us how these are to be defined when they target outcomes of deliberation as opposed to targeting internal reasoning processes. In some other work they attempt to tackle this by focusing on argument reception processes, giving a descriptive account of how outcomes are influenced by various conditions placed on deliberation. (I have some references I think...). This suggests characterizing social rationality in terms of such conditions. Perhaps even the conditions for deliberation are even more important than the perceived "classically correct" nature of the outcome? We don't now, but we propose formal logic as a means for further study of such questions....}

\section{Claims}

\begin{itemize}
\item Without strategic cooperation, any collection of views can be represented by two extremes.
\item No one's beliefs can be enforced in a deliberative scenario where every participant disagrees about everything with everyone else.
\item In order to make your own belief a possible outcome of deliberation, you should disagree with everyone. In this case, you make \emph{any} belief a possible outcome.
\item In order to enforce your own belief, you need to agree with something that everyone already agrees with (winning an argument requires building on the common ground)
\item In order to enforce a belief not already enforced, you need to disagree with something that everyone else agrees with (originality requires dissent)
\item The \emph{ability} of an agent/coalition who knows what others believe is defined by picking a point between the already established common ground and arbitrariness, making more things possible.
\item The ability of an agent who does not know what others believe is to establish make it impossible 
\end{itemize}
\section{Background on abstract argumentation}\label{sec:abt}

From a formal logical perspective, abstract argumentation theory concerns the following question:
Given some collection of arguments $\Pi$ and some model of their content, what statements can we form regarding their interaction, and how do we judge which such statements we should accept? This, in particular, is the question we address below, and we will follow \cite{dung} in relying on argumentation frameworks (AFs) as models. An AF over $\Pi$ is simply a directed graph $E \subseteq \Pi \times \Pi$ where edges $(p,q) \in E$ are thought of as attacks, with $p$ attacking $q$ in this case. We use the notation $E^+(x) = \{y \in \Pi \mid (x,y) \in E\}, E^-(x) = \{y \in \Pi \mid (y,x) \in E\}$, extended to sets such that $E^\ast(A) = \bigcup_{x \in A}E^\ast(x)$ for $\ast \in \{+,-\}$. For a given AF $E$, we will use $\Pi(E) = \{x \mid E^+(x) \cup E^-(x) \not = \emptyset\}$ to denote the set of arguments from $\Pi$ which appear in some attack from $E$. We will often defined AFs simply by depicting the attacks they contain, as in Figure \ref{fig:1}. \\
\begin{figure}
$$
\xymatrix{p \ar@(lu,ld) \ar@/_/[r] & q \ar@/_/[l] \ar@/_/[r] & q' \ar@/_/[l] \ar@/_/[r] & p' \ar@(ru,rd) \ar@/_/[l] }
$$
\caption{An AF $E$ such that $\Pi(E) = \{p,q,q',p'\}$}
\label{fig:1}
\end{figure}

The logical language we will use is a simple propositional language $\lang$ with implication, defined by the following grammar.
$$
\phi := p \ \mid \ \neg \phi \ \mid \ \phi \to \psi \
$$
where $p \in \Pi$. Similarly to the notational convention we employ for AFs, we will use $\Pi(\phi)$ to denote the atoms appearing in a formula $\phi$. We note that since $\lang$ is a finitary language (not involving any infinite connectives), $\Pi(\phi)$ is always finite. Intuitively, we may think about formulas from $\lang$ as meta-arguments concerning how we think various arguments \emph{should} fare according to various semantics that may be formulated using Dung's model. Indeed, $\lang$ is essentially a \emph{modal} language when interpreted on argumentation frameworks. For instance, the argument $p$ may be read as the argument that $p$ \emph{should} be accepted, but a different (modal) construction will be needed in order to express the argument that $p$ \emph{must} be accepted. We will see this more clearly later, after we have defined the semantics, but we notice already now how it ensures, intuitively speaking, that $p$ is read as a meta-claim that it is reasonable to ascribe to any argument, regardless of how it is formed. Indeed, the single logical property we think it is safe to say that all arguments share, is the fact that they all make the implicit assertion that they should be accepted.

The formulas of $\lang$ will be interpreted under a range of different semantics that have been applied to AFs in the existing literature on argumentation. For a logical inquiry, the level of abstraction provided by such semantics will be helpful, as we want to talk about properties of argumentation as such, independent of any particular domain or field of application. On the other hand, the concern is sometimes voiced that the view of argumentation offered by such semantics results in theories of argumentation that are  \emph{too} abstract and simplifying, and not sufficiently expressive to provide useful and realistic insights. Hence, some argue, there is a need to move beyond the simple graph-based semantics to consider more actively the underlying scenarios that may instantiate particular AFs.

We agree that such research is also interesting, but we note that the claim that argumentation semantics are too abstract to be of much use in themselves is itself a claim which it is difficult to address properly without first developing a clear logical foundation, which allows us to identify the validities that argumentation semantics give rise to. The set of validities, in particular, contain those facts about these semantic that remain true in \emph{all} situations that the model is capable of representing. Hence the validities also encode its expressive power. For instance, if the set of validities is deemed reasonable, for instance because they are all generated from a reasonable set of axioms and inference rules, it counts as strong evidence that using AFs to model argumentation is itself reasonable, in the sense that whatever it does not count as an argumentative scenario \emph{should} be disregarded. On the other hand, if the set of validities is deemed inappropriate, then it provides critics with a good \emph{formal} argument against the appropriateness of using AFs.\footnote{Moreover, for a nice illustration of how initial intuitions can be misleading in such matters, we point to the notion of an \emph{abstract dialectical} framework introduced in \cite{ab}. It is in some sense a \emph{vast} generalization of Dung's framework, allowing such things as supporting arguments, weighting of arguments, and more. However, as shown in \cite{abcon}, it is possible to simulate the behavior of any abstract dialectical framework using a regular AF, meaning that at the logical level there is a correspondence allowing us to replace dialectical frameworks by equivalent AFs. The difference concerns only how the information is presented (which is of course still relevant in practical applications, as well as in the context of designing efficient exact algorithms, although \cite{abcon} also proves that the translation has polynomial cost).}

Before we move on, we also remark that there has recently been quite a lot of work devoted to finding neat ways to \emph{define} argumentation semantics using formal logic, see for instance \cite{Grossi1,Grossi2} which relies on modal logic, and \cite{Arieli} which uses quantified boolean formulas.\footnote{For completeness, we also mention \cite{GabModGr,Gablog} which develops similar ideas by exploiting (other) ways in which to define argumentation semantics in modal logic, and \cite{Gablog2}, which relates argumentation to three-valued labellings for logic programing, and can hence be viewed as a leading up to both the work in this paper and in \cite{Arieli}.}

While we think this work is interesting, we do not focus on defining argumentation semantics in this paper, but rather on developing logics that allow us to reason \emph{about} them, and to study their meta-logical properties. Certainly, this can also be done with the aid of formal logics which are expressive enough to define all these semantics in the object-language, but such an approach easily runs the risk of complicating matters to the extent that interesting meta-logical results become hard or impossible to obtain. In particular, it will typically require us to axiomatize fragments of very powerful logics that may not admit any straightforward axiomatization, if they are at all decidable. To the best of out knowledge, there has so far not been any work carried out in this vein which has succeeded in establishing results that serve to axiomatize the validities or notions of logical consequence arising from argumentation semantics.

Lastly, let us briefly address the possible worry that AFs are themselves \emph{too} expressive, that they allow us to represent scenarios that should not count as legitimate models, either because they \emph{will not} arise in some context of application, or else because they \emph{should not} arise because they represent ``incorrect" argumentation. It might be, for instance that they can only be instantiated by argumentation scenarios where participants use fallacious arguments, or else by arguments that contradict facts or logical truths. If such AFs are identified and excluded, then this \emph{might} (but need not) imply that Dung's theory is too permissive, and that more validities should obtain. However, even if this is the case, it need not be taken as an argument against the theory. Rather, it can be taken to raise yet more questions regarding its meta-logical properties. For instance, if we can show that there is a strongly complete axiomatization of Dung's theory, it is easy to incorporate the view that more validities should be admitted to restrict the scenarios under consideration; we may simply add them as axioms and continue to reason in exactly the same way as before.

On a related note, we remark that while research on instantiated argumentation in AI have proposed limiting principles regarding \emph{how} AFs should be instantiated underlying theories describing the content of arguments, none of these limits seem to entail that there are AFs which \emph{do not instantiate} a legitimate argumentative scenario. Hence at the level of abstraction of which Dung's theory is formulated, no new validities at the level of pure AFs appears to arise from this work, see e.g., \cite{rational,modaspic+}. This, in any event, is true as far as the \emph{rationality postulates} from \cite{rational} are concerned. They are generally formulated principles describing reasonable ways to instantiate an AF using an underlying theory of default logic, and a consensus seems to have emerged that it is  a blueprint for what counts as an adequate instantiated theory, see e.g., \cite{modaspic+}. It is not hard to see, however, that even with the rather restrictive instantiation theories considered in \cite{rational}, all finite AFs will emerge as the instantiation of \emph{some} theory satisfying the postulates.\footnote{Given a finite AF $E$, it is enough to consider the default theory which contains the rules $\bigcup_{x\in \Pi}\band_{y \in E^+(x)}\{x \Rightarrow \neg y\}$, such that for all $x$, all rules with $x$ on the left are labeled by $x$ (meaning that the rules with $x$ on the left instantiates attacks on all the arguments $y$ that they are meant to attack according to $E$)}

Having briefly presented the basics and further motivated our logical approach, we now move on to formally define some of the various semantics that have been proposed for AFs, and the corresponding logics that arise when we put them to use in order to interpret $\lang$.

\subsection{Logics for skeptical reasoning about argumentation}\label{subsec:1}


Given an AF $E$, the task of an argumentation semantics is to identify sets of arguments that can be held successfully together. Typically, this involves various formalizations of the intuition that the set should be internally consistent and able to defend itself against attack from other arguments. Different semantics differ about the details, but they all share the same overall aim, which is to answer, for any $p \in \Pi$, whether $p$ should be accepted in the argumentative scenario described by $E$. Hence different semantics all have the same signature; they are defined as an operator $\sem$ which takes an AF $E$ and returns a set of sets of $\sem(E) \subseteq 2^\Pi$ -- those sets that are \emph{acceptable} according to $E$. Moreover, to the best of our knowledge, all semantics that have been considered reasonable share the property that that arguments in an acceptable set should be internally consistent, free of internal conflict. Formally, for all such semantics $\sem$, all AFs $E$ and all $A \in \sem(E)$, we have $E^-(A) \subseteq \Pi \setminus A$; no two arguments in $A$ attack each other.

It might be tempting to think that argumentation semantics should give rise to a binary notion of acceptance; for a given argument, it is accepted or it is not. However, a moments thought will show that this perspective fails to do justice to the nature of the structure $(E,\sem)$ in two important ways. First, there is the question of whether it is correct to say that $p$ is accepted on $E$ under $\sem$ when there \emph{exists} some $A \in \sem(E)$ such that $p \in A$, or whether we should require $p \in A$ for \emph{all} such $A$. Both notions of acceptance have been studied, and the former is typically dubbed \emph{credulous} while the latter is referred to as \emph{skeptical}. In this paper we will address the logic of skeptical acceptance, but we regard the development of logics for addressing credulous reasoning, and particularly the relationship between skeptical and credulous acceptance, as an important goal for future research.\footnote{It is natural to view skeptical and credulous acceptance as dual \emph{modalities}, in particular, suggesting the study of the set of validities characterizing their interaction.} 

The second sense in which acceptance is not a binary notion has to do specifically with the structure of $E$. In particular, given any $A \in \sem(E)$ the status of $p$ with respect to $A$ can be any of the following:
\begin{equation}\label{eq:status}
\begin{array}{lr}
1: p \in A & 2: p \in E^+(A) \\ 3: p \in \Pi \setminus (A \cup E^+(A)) 
\end{array}
\end{equation}
Notice that by conflict-freeness of $A$, it follows that if $p \in E^+(A)$ then $p \not \in A$. Hence when the focus is on the status of individual arguments, we might as well view $\sem(E)$ as a set of partitions of $\Pi$ into three disjoint sets or, equivalently, as a collection of so called \emph{(Caminada) labellings}, functions $\clab: \Pi \to \three$
such that for all $x \in \Pi$:
\begin{equation}\label{eq:cam}
\begin{array}{l}
\clab(x) = 0 \iff \exists y \in E^-(x): \clab(y) = 1
\end{array}
\end{equation}
For any AF $E$ we let $\proto(E)$ be the set of all labellings for $E$, and we define $\clab^1 = \{x \in \Pi \mid \clab(x) = 1\}, \clab^0 = \{x \mid \clab(x) = 0\}$ and $\clab^{\frac{1}{2}} = \{x \in \Pi \mid \clab(x) = \frac{1}{2}\}$. This, in particular, defines a semantics for argumentation such that for all $E$, we regard $A \subseteq \Pi$ as acceptable if there is some $\clab \in \proto(E)$ such that $\clab^1=A$.\footnote{Hence it is not hard to see that values assigned by labellings correspond to the three points of Equation \ref{eq:status} whenever we restrict attention to conflict-free sets of accepted arguments. Notice, in particular, that $p \in \clab^0 \iff p \in E^+(\clab^1)$ and $p \in \clab^{\frac{1}{2}} \iff p \in \Pi \setminus (\clab^1 \cup \clab^0)$} In applications of argumentation theory, this is usually considered too permissive, and a range of various restrictions has been considered, each giving rise to a new semantics, the most well-known of which are defined in Figure \ref{fig:argsem}.

\begin{figure}
$\begin{array}{ll}
\text{\footnotesize{Admissible: }} & a(E) = \{\clab \in \proto(E) \mid  E^-(\clab^1) \subseteq \clab^0\} \\
\text{\footnotesize{Complete:}} & c(E) =  \{\clab \in \proto(E) \mid \\ & \hspace{1.2cm} \clab^1 = \{x \in \Pi \mid E^-(x) \subseteq \clab^0\}\} \\
%\forall x \in \Pi: \\ & \clab(x) = 1 \iff \forall y \in E^-(x): \clab(y) = 0\} \\
\text{\footnotesize{Grounded:}} & g(E) = \{\bigcap c(E)\} \\
\text{\small{Preferred:}} \ \ & p(E) = \{\clab_1 \in a(E) \mid \forall \clab_2 \in a(E): \clab^1_1 \not \subset \clab^1_2\} \\
\text{\small{Semi-stable:}} \ \ & ss(E) = \{\clab_1 \in a(E) \mid \forall \clab_2 \in a(E): \clab^{\frac{1}{2}}_1 \not \supset \clab^{\frac{1}{2}}_2\} \\
\text{\small{Stable:}} \ \ & s(E) = \{\clab \in a(E) \mid \clab^{\frac{1}{2}} = \emptyset\} 
\end{array}$
\caption{Various semantics, defined for any $E \subseteq \Pi \times \Pi$}
\label{fig:argsem}
\end{figure}

Following Dung \cite{Dung} we represent argumentation structures by directed graphs $\af = (S,E)$, such that $S$ is a set of arguments and $E \subseteq S \times S$ encode the attacks between them, i.e., such that if $(x,y) \in E$ then the argument $x$ attacks $y$. Traditionally, most work in formal argumentation theory has focused on defining and investigating notions of successful sets of arguments, in a setting where the argumentation framework is given and remains fixed. Such notions are typically formalized by an \emph{argumentation semantics}, an operator $\sem$ which returns, for any AF $\af$, the set of sets of arguments from $\af = (S,E)$ that are regarded as successful combinations, i.e., such that $\sem(\af) \subseteq 2^S$. Many proposals exists in the literature, we point to \cite{BaroniEval} for a survey and formal comparison of different semantics. While some semantics, such as the grounded and ideal semantics, return a unique set of arguments, the ``winners'' of the argumentation scenario encoded by $\af$, most semantics return more than one possible collection of arguments that \emph{would} be successful if they were held together. For instance, the admissible semantics, upon which many of the other well-known semantics is built, returns, for each AF $\af$, the following sets of arguments:

$$
a(\af) = \{A \subseteq S \mid E^-(A) \subseteq E^+(A) \subseteq S \setminus A\}
$$
That is, the admissible sets are those that can defend themselves against attacks (first inclusion), and do not involve any internal conflicts (second inclusion). A strengthening that is widely considered more appropriate (yet incurs some computational costs) is the \emph{preferred} semantics $\sf p$, which is defined by taking only those admissible sets that are set-theoretically maximal, i.e., such that they are not contained in any other admissible set. In general, an AF admits many preferred sets, and even more admissible ones. Indeed, notice that the empty set is always admissible by the default (the inappropriateness of which provides partial justification for using preferred semantics instead). As a simple example, consider $\af$ below.
$$\begin{array}{ll}
\af: \xymatrix{p \ar@/_/[r] & q \ar@/_/[l] } &\hspace{4em} a(\af) = \{\emptyset,\{p\},\{q\}\}, \ \sf p(\af) = \{\{p\},\{q\}\}\}
\end{array}
$$

Indeed, it seems hard to say which one of $p$ and $q$ should be regarded as successful in such a scenario. In the absence of any additional information, it seems safest to concede that choosing either one will be a viable option. Alternatively, one may take the view that due to the undetermined nature of the scenario, it should not be permitted to regard either argument as truly successful. This, indeed, is the view taken by unique status semantics, such as the grounded and ideal semantics. However, while such a restrictive view might be appropriate in some circumstance, it seems unsatisfactory for a general theory of argumentation. Surely, in most real-world argumentation situations, it is not tenable for an arbitrator to refrain from making a judgment whenever doing so would involve some degree of discretion on his part.

Since argumentation semantics typically only restrict the choice of successful arguments, without determining it completely, a modal notion of \emph{acceptance} arises, usually referred to as \emph{skeptical} acceptance in argumentation parlor, whereby an argument is said to be skeptically accepted by $\af$ under $\sem$ if $\forall S \in \sem(\af): p\in S$. The dual notion is called \emph{credulous} acceptance, and obtains just in case $\exists S \in \sem(\af): p\in S$. Moreover, since the choice among elements of $\sem (\af)$ can itself be a contentious issue, and is not one which can be satisfactorily resolved by single-agent argumentation theory, there has been research devoted to giving an account of multi-agent interaction concerning the choice among members of $\sem(\af)$, see \cite{manipulation,lying}. While this is interesting, it seems that another aspect of real-world argumentation has an even stronger multi-agent flavor, namely the process by which one arrives at a common AF in the first place. Certainly, two agents, $a$ and $b$, might disagree about whether to choose $p$ or $q$ in $\af$ considered above, but as it stands, such a choice appears arbitrary and, most likely, the two agents would also be willing to admit as much. Arguably, then, the disagreement itself is only superficial. The agents disagree, but they provide no \emph{reason} for their different preferences, and do bot provide any content or structure to substantiate them. This leaves an arbitrator in much the same position as he was in before: he might note the different opinions raised, but he has no basis upon which to inquire into their merits, and so his choice must, eventually, still be an exercise in discretion.

In practice, however, it would have to be expected that if the agents $a$ and $b$ were really committed to their stance, they would not simply accept that $\af$ correctly encodes the situation and that the choice is in fact arbitrary. Rather, they would produce \emph{arguments} to back up their position. It might be, for instance, that agent $a$, who favors $p$, claims that $q$ is inconsistent for some reason, while agent $b$, who favors $q$, makes the same accusation against the argument $p$. Then, however, we are no longer justified in seeing this as disagreement about which choice to make from $\sem(\af)$. Rather, the disagreement concerns the nature of the argumentation structure itself. The two agents, in particular, put forth different \emph{views} on the situation. For instance, in our toy example, we would have to consider the following two AFs, where $V_a, V_b$ encode the views of $a$ and $b$ respectively.

\begin{equation}\label{ex}
\begin{array}{ll}
V_a: \hspace{2em} \xymatrix{p \ar@(lu,ld) \ar@/_/[r] & q \ar@/_/[l] } & \hspace{4em} V_b: \xymatrix{p \ar@/_/[r] & q \ar@(ru,rd) \ar@/_/[l] }
\end{array}
\end{equation}

Then the question arises: what are we to make of this? 

In the following, we address this question, and we approach it from the conceptual starting point that evaluating (higher-order) differences of opinion such as that expressed by $V_a,V_b$ takes place iteratively, through a process of \emph{deliberation}, leading, in a step-by-step fashion, to an aggregated \emph{common} $\af$. Such a process might be instantiated in various ways: it could the agents debating the matter among themselves and reaching some joint decision, or it could be an arbitrator who considers the different views and reasons about them by emulating such a process. Either way, we are not interested in attempting to provide any guidance towards the ``correct'' outcome, which is hardly possible in general. Rather, we are interested in investigating the modalities that arise when we consider the space of all possible outcomes (where possible will be defined in due course). Moreover, we are interested in investigating structural questions, asking, for instance, about the importance of the order in which arguments are considered, and the consequences of limiting attention to only a subset of arguments.

We use a dynamic modal logic to facilitate this investigation, and in the next section we define the basic framework and show that model checking is decidable even on infinite AF's, as long as the agent's views remains finitely branching, i.e., as long as no argument is attacked by infinitely many other arguments. We will parameterize our logic by an argumentation semantics, so that it can be applied to any such semantics which satisfies a normality condition. In particular, let $C(\af) = \{C^{\af}_1,\ldots,C^{\af}_i,\ldots\}$ denote the (possibly infinite) set of maximal connected components from $\af$ (the set of all maximal subsets of $S$ such that any two arguments in the same set are connected by a sequence of attacks). Then we say that a semantics $\sem$ is \emph{normal} if we have, for any $\af = (S,E)$

\begin{equation}\label{eq:normal}
A \in \sem(\af) \Leftrightarrow A = \bigcup_{i}A_i \text{ for some } A_1,\ldots,A_i,\ldots \text{ s.t. } A_i \in \sem(C^{\af}_i) \text { for all } i
\end{equation}
That is, a semantics is normal if the status of an argument depends only on those arguments to which it has some (indirect) relationship through a sequence of attacks. We remark that all argumentation semantics of which we are aware satisfies this requirement, hence we feel justified in dubbing it normality.


\section{Deliberative dynamic logic}\label{sec:ddl}

We assume given a finite non-empty set $\agents$ of agents and a countably infinite set $\Pi$ of arguments.\footnote{Possibly ``statements'' or ``positions'', depending on the context of application.} The basic building block of dynamic deliberative logic is provided in the following definition.

\begin{definition}\label{def:basis} A \basis for deliberation is an $\agents$-indexed collection of digraphs $\views = (V_a)_{(a \in \agents)}$, such that for each $a \in \agents$, $V_a \subseteq \Pi \times \Pi$.
\end{definition}

Given a basis which encodes agents' view of the arguments, we are interested in the possible ways in which agents can deliberate to reach \emph{agreement} on how arguments are related. That is, we are interested in the set of all AFs that can plausibly be seen as resulting from a \emph{consensus} regarding the status of the arguments in $\Pi$. What restrictions is it reasonable to place on a consensus? It seems that while many restrictions might arise from pragmatic considerations, and be implemented by specific protocols for ``good'' deliberation in specific contexts, there are few restrictions that can be regarded as completely general. For instance, while there is often good reason to think that the position held by the majority will be part of a consensus, it is hardly possible to stipulate an axiomatic restriction on the notion of consensus amounting to the principle of majority rule. Indeed, sometimes deliberation takes place and leads to a single dissenting voice convincing all the others, and often, these deliberative processes are far more interesting than those that transpire along more conventional lines. However, it seems reasonable to assume that whenever \emph{all} agents agree on how an argument $p$ is related to an argument $q$, then this relationship is part of any consensus. This, indeed, is the only restriction we will place on the notion of a consensus; that when the AF $\af$ is a consensus for $\basis$, it must satisfy the following \emph{faithfulness} requirement.
\begin{itemize}
\item \emph{For all $p,q \in \Pi$, if there is no disagreement about $p$'s relationship to $q$ (attack/not attack), then this relationship is part of $\af$}
\end{itemize}

This leads to the following definition of the set $\cons \views$, which we will call the set of \emph{complete assents} for $\views$, collecting all AFs that are faithful to $\views$.

\begin{equation}\label{def:consensus}
\cons \views = \left\{\af \subseteq \Pi \times \Pi ~\left|~ \bigcap_{a \in \agents}V_a \subseteq \af \bigcup_{a \in \agents}V_a\right.\right\}
\end{equation}

An element of $\cons \views$ represents a possible consensus among agents in $\agents$, but it is an \emph{idealization} of the notion of assent, since it disregards the fact that in practice, assent tends to be \emph{partial}, since it results from a dynamic process, emerging through \emph{deliberation}. Indeed, as long as the number of arguments is not bounded we can \emph{never} hope to arrive at complete assent via deliberation. We can, however, initiate a process by which we reach agreement on more and more arguments, in the hope that this will approximate some complete assent, or maybe even be \emph{robust}, in the sense that there is \emph{no} deliberative future where the results of current partial agreement end up being undermined. Complete assent, however, arises only in the limit.

When and how deliberation might successfully lead to an approximation of complete assent is a question well suited to investigation with the help of dynamic logic. The dynamic element will be encoded using a notion of a deliberative event -- centered on an argument -- such that the set of ways in which to relate this arguments to arguments previously considered gives rise to a space of possible deliberative time-lines, each encoding the continued stepwise construction of a joint point of view. This, in turn, will be encoded as a monotonically growing AF $\af = (S,E)$ where $S \subseteq \Pi, E \subseteq S \times S$ and such that faithfulness is observed by all deliberative events. That is, an event consists in adding to $\af$ the agents' combined view of $p$ with respect to the set $S \cup \{p\}$. This leads to the following collection of possible events, given a basis $\views$, a partial consensus\footnote{These ``partial consensuses'' are sometimes referred to as ``contexts'' when they are used to describe graphs inductively, as we will do later.} $\af = (S,E)$ and an argument $p \in \Pi$:

\begin{equation}\label{eq:update}
\update \views \af p = \left\{X ~\left|~ \bigcap_{a \in \agents}\restr {V_a} {S \cup \{p\}} \subseteq X \subseteq \bigcup_{a \in \agents}\restr {V_a}{S \cup \{p\}}\right.\right\}
\end{equation}

To provide a semantics for a logical approach to deliberation based on such events, we will use Kripke models.

\begin{definition}[Deliberative Kripke model]\label{def:main} Given an argumentation semantics $\sem$ and a set of views $\views$, the deliberative Kripke models induced by $\views$ and $\sem$ is the triple $\kmod \views \sem = (\carriers \views, \rels \views,\pis \sem)$ such that
\begin{itemize}
\item $\carriers \views$, the set of points, is the set of all pairs of the form $q = (q_S,q_E)$ where $q_S \subseteq \Pi$ and $$\bigcap_{a \in \agents}\restr {V_a} {q_S} \subseteq q_E \subseteq \bigcup_{a \in \agents}\restr {V_a} {q_S}$$
The basis $\views$ together with our definition of an event, given in Equation \ref{eq:update}, induces the following function, mapping states to their possible deliberative successors, defined for all $p \in \Pi, q \in \carriers \views$ as follows
$$succ(p, q) \quad := \quad \{~(q_S \cup \{p\}, q_E \cup X) ~|~ X \in \update \views q p~\}$$ 
We also define a lifting, for all states $q \in \carriers \views$:
$$succ(q) \quad := \quad \{~q' \mid \exists p \in \Pi: q' \in succ(q,p)\}$$
\item $\rels \views: \Pi \cup \{\exists\} \to 2^{\carriers \views \times \carriers \views}$ is a map from symbols to relations on $\carriers \views$ such that 
\begin{itemize} \item $\rels \views(p) = \{(q,q') \mid q' \in succ(p,q)\}$ for all $p \in \Pi$ and
\item $\rels \views(\exists) = \{(q,q') \mid q' \in succ(q)\}$,
\end{itemize} \vspace{1em}
\item $\pis \sem: \carriers \views \to 2^{(3^\Pi)}$ maps states to labellings such that for all $q \in \carriers \views$ we have $\pis \sem(q) = (\pi_1,\pi_0,\pi_{\frac{1}{2}})$ with $$\pis \sem(q) = \{\pi \mid \pi_1 \in \sem(q), \pi_0 = \{p \in q_S \mid \exists q \in \pi_1: (q,p) \in q_E\}\}$$
\end{itemize}
\end{definition}

Notice that in the last point, we essentially map $q$ to the sets of extensions prescribed by $\sem$ when $q$ is viewed as an AF. We encode this extension as a three-valued labeling, however, following \cite{caminada06}. Notice that the default status, attributed to all arguments not in $q_S$, is $\frac{1}{2}$. The logical language we will use consists in two levels. For the lower level, used to talk about static argumentation, we follow \cite{Arieli,Sjur-SYNT} in using {\L}ukasiewicz three-valued logic. Then, for the next level, we use a dynamic modal language which allows us to express consequences of updating with a given argument, and also provides us with existential quantification over arguments, allowing us to express claims like ``there is an update such that $\phi$''. This leads to the language $\dlangm$ defined by the following \acro{bnf}'s

$$ \phi \quad ::= \cdia \alpha ~|~ \neg \phi ~|~ \phi \wedge \phi ~|~ \ddia p \phi ~|~ \adia \phi$$ 
where $p \in \Pi$ and $\alpha \in \lblack$ where $\lblack$ is defined by the following grammar:
$$
\alpha ::= p \ | \ \neg \phi \ | \ \phi \to \phi $$
for $p \in \Pi$.

We also use standard abbreviations such that $\abox \phi = \neg \adia \neg \phi$, $\dbox p \phi = \neg \ddia p \neg \phi$ and $\cbox \alpha = \neg \cdia \neg \alpha$. We also consider that standard boolean connectives abbreviated as usual for connectives not occurring inside a $\cdia$-connective and abbreviations for connectives of {\L}ukasiewicz logic in the scope of $\cdia$-connectives.

Next we define truth of formulas on deliberative Kripke models. We begin by giving the valuation of complex formulas from $\lblack$, which is simply three-valued {\L}ukasiewicz logic.

\begin{definition}[$\alpha$-satisfaction] For any three-partitioning $\pi = (\pi_1,\pi_0,\pi_{\frac{1}{2}})$ of $\Pi$, we define 
\begin{align*}
\overline \pi(p) &= x \text{ s.t } p \in \pi_x \\
\overline\pi(\neg\alpha) &= 1 - \overline\pi(\alpha)\\
\overline\pi(\alpha_1 \to \alpha_2) &= \min \{1, 1 - (\overline\pi(\alpha_1) - \overline\pi(\alpha_2))\} 
\end{align*}
\end{definition}

Now we can give a semantic interpretation of the full language as follows.

\begin{definition}[$\dlangm$-satisfaction]\label{def:ddlm}
Given an argumentation semantics $\sem$ and a basis $\views$, truth on $\kmod \views \sem$ is defined inductively as follows, in all points $q \in \carriers \views$.
\begin{align*}
\kmod \views \sem, q \vDash \cdia \alpha & \iff \quad \text{there is } \pi \in \pis \sem(q) \text{ s.t. } \overline\pi(\phi) = 1 \\
\kmod \views \sem, q \vDash \neg \phi \quad & \iff \quad \text{not } \kmod \views \sem, q \vDash \phi \\
\kmod \views \sem, q \vDash \phi \wedge \psi & \iff \quad  \text{both } \kmod \views \sem, q \vDash \phi \text{ and } \kmod \views \sem, q \vDash \psi \\
\kmod \views \sem, q \vDash \ddia \pi p & \iff \quad \exists (q,q') \in \rels \views(p): \kmod \views \sem,q' \vDash \phi \\
\kmod \views \sem, q \vDash \adia \phi & \iff \quad \exists (q,q') \in \rels \views(\exists): \kmod \views \sem,q' \vDash \phi \\
\end{align*}
\end{definition}

%We will often formulate propositions about our models with respect to some point, i.e., by making reference to pointed models. When the models are pointed, we will make use of a useful abstraction which will facilitate more %succinct proofs. Often, but not always, the models will be rooted in the empty graph. The point $(\emptyset, \emptyset)$ in the above definition of a model. 

To illustrate the definition, we return to the example depicted in (\ref{ex}). In Figure \ref{tbl:dynamism-3}, we depict this basis together with a fragment of the corresponding Kripke model, in particular the fragment arising from the $p$-successors of $(\emptyset,\emptyset)$.

%
%\begin{figure}[ht]
%  \centering
%  \def\svgwidth{\textwidth}
%  \begin{tiny}
%    \input{img/example.pdf_tex}
%  \end{tiny}
%  \caption{Almost complete tree.}
%  \label{tbl:dynamism-3}
%\end{figure}

\begin{figure}[ht]
\begin{minipage}[t]{10em}
$$ \small
\views = \left\{\begin{array}{l} V_a: \hspace{1.5em} \xymatrix{p \ar@(lu,ld) \ar@/_/[r] & q \ar@/_/[l] } \vspace{2em}  \\ V_b: \xymatrix{p \ar@/_/[r] & q \ar@(ru,rd) \ar@/_/[l] }
\end{array}\right\}$$
\end{minipage}
\begin{minipage}[t]{28em} \hspace{0.2em}
  \centering
  \def\svgwidth{\textwidth}
  \begin{tiny}
%    \input{img/example2.pdf_tex}
  \end{tiny}
\end{minipage}
  \caption{A fragment of the deliberative Kripke model for $\views$.}
  \label{tbl:dynamism-3}
\end{figure}

Let us assume that $\sem = \sf p$ is the preferred semantics. Then the following list gives some formulas that are true on $\kmod \views \sem$ at the point $(\emptyset,\emptyset)$, and the reader should easily be able to verify them by consulting the above fragment of $\kmod \views \sem$.

$$
\begin{array}{lll}
\ddia p \cbox p, & \adia \cbox p, & [p] \adia \cbox q, \\
\neg [p] \adia \cdia p, & \ddia p \adia \cbox \neg p, & \adia \adia (\cdia p \land \cdia q)
\end{array}
$$

We can also record some validities that are easy to verify against Definition \ref{def:main}. 

\begin{proposition}\label{prop:val}
The following formulas are all validities of $\dlangm$, for any $p,q \in \Pi$, $\phi \in \dlangm$. \begin{enumerate}
\item $\ddia p \ddia q \phi \leftrightarrow \ddia q \ddia p \phi$
\item $\ddia p \dbox q \phi \to \dbox q \ddia p \phi$
\item $\adia \abox \phi \to \abox \adia \phi$
\item $\ddia p \ddia p \phi \to \ddia p \phi$ 
\end{enumerate}
\end{proposition}

We remark that $\dbox q \ddia p \phi \to \ddia p \dbox q \phi$ is \emph{not} valid, as witnessed for instance by the following basis $\views$, for which we have $\kmod \views p, (\emptyset,\emptyset) \models \dbox q \ddia p \cbox p$ but also $\kmod \views p, (\emptyset,\emptyset) \models \dbox p \ddia q \cbox q$ (as the reader may easily verify by considering the corresponding Kripke model).

$$
\views = \left\{\begin{array}{ll}V_a: \xymatrix{p \ar[r] & q} \\ V_b: \xymatrix{p & q \ar[l]} \end{array}\right\}$$

Finally, let us notice that as $\Pi$ is generally infinite, we must expect to encounter infinite bases. This means, in particular, that our Kripke models are often infinite. However, in the next section we show that as long as $\views$ is \emph{finitary}, meaning that no agent $a \in \agents$ has a view where an argument is attacked by infinitely many other arguments, we can solve the model-checking problem also on infinite models.

\section{Model checking on finitary models}\label{sec:mcheck}

Towards this result, we now introduce some notation and a few abstractions to simplify our further arguments. We will work with labeled trees, in particular, where we take a tree over labels $X$ to be some
 non-empty, prefix-closed subset of $X^*$ (finite sequences of elements of $X$). Notice that trees thus defined contain no infinite sequences. This is intentional, since we will ``shrink'' our models (which may contain infinite sequences of related points), by mapping them to trees. To this end we will use the following structures. 

\begin{definition}\label{def:iaf} Given a basis $\views$, we define $\iterate \views$, a set of sequences over $\Pi \times 2^\Pi$ labeled by AFs, defined inductively as follows
\begin{description}
\item [Base case:] $\epsilon \in \iterate \views$ and is labeled by the AF $\af(\epsilon) = (S(\epsilon),E(\epsilon))$ where $S(\epsilon) = \emptyset = E(\epsilon)$.
\item [Induction step:] If $x \in \iterate \views$, then for any $p \in \Pi$ and any partial assent $X = \update \views x p$, we have $x;(p, X) \in \iterate \views$ labeled by the AF $\af(x;(p,X))$ where $S(x;(p, X)) = S(x) \cup \{p\}$ and $E(x;(p, X)) = E(x) \cup X$.
\end{description}
%Infinite IAFs are infinite sequences of the form $x_1;x_2;x_3;\dots$ such that, for all $i \geq 1$, there is $p \in \Pi$ and an $X = \update \views {(S(x_i),E(x_i))} p$ such that $x_{i+1} = (p, X)$. An argument $p \in \Pi$ is in the %argument set of this IAF, that is $p \in S(x_1;x_2;x_3;\dots)$ if, and only if, there is an $i$ such that $x_i = (p, \_)$ (second component is irrelevant). (Similarly for edges.)
\end{definition}
%The labels we will be interested in are representations of applications of $\sem$ in any given state. Let $\emptyset \subseteq V \subseteq \bigcup_{a \in \agents} V_a$ be some graph. All graphs actually generated with our %construction will be included here, but not neccessarily conversely. 
To adhere to standard naming we use $\epsilon$ to denote the empty string. It should not be confused with the argumentation semantics $\sem$. This will also be clear from the context.
We next define tree-representations of our Kripke models.

\begin{definition}\label{def:treerep} Let $\kmod \views \sem$ be some model. The \emph{tree representation} of $\kmod \views \sem$ is the set $T$, together with the representation map $\gamma: \carriers \views \to 2^T$, defined inductively as follows
\begin{description}
\item[Base case] $\epsilon \in T$ is the root with $\gamma((\emptyset, \emptyset)) = \{\epsilon\}$.
\item[Induction step] For any $x \in T, q \in \kmod \views \sem$ with $x \in \gamma(q)$ and $q' \in succ(q)$ witnessed by $p \in \Pi$ and $X \in \update \views x p$, we have $x;(p, X) \in T$ with $q' \in \gamma(x;(p,X))$.
%\item[Closure] $(T,\gamma)$ is the pair consisting of the smallest $T,\gamma$ which satisfies the above.
\end{description}
\end{definition}

Notice that the tree-representation is a tree where each node is an element of $\iterate \views$. Some single states in $\kmod \views \sem$ will have several representations in a tree. That is, $\gamma(q)$ may not be a singleton. On the other hand, it is easy to see that for every state $q \in \kmod \views \sem$, and every path from $(\emptyset, \emptyset)$ to $q$, there will be a node $x \in T$ such that $q \in \gamma(x)$.

The main result of our paper is that model checking $\dlangm$-truth at $(\emptyset,\emptyset)$ is tractable as long as all views are \emph{finitely branching}, i.e., such that for all $a \in \agents, p \in \Pi$, $p$ has only finitely many attackers in $V_a$. Clearly this requires shrinking the models since the modality $\adia$ quantifies over an infinite domain whenever $\Pi$ is infinite. We show, however, that attention can be restricted to arguments from $\Pi$ that are \emph{relevant} to the formula we are considering. To make the notion of relevance formal, we will need the following measure of complexity of formulas.
%
%We now consider the problem of verifying that a model $K = \langle \carriers \views, R, \sem \rangle$ based on $\views$ satisfies a formula $\phi \in \langa$. Both $K$ and $\views$ might be infinite, and ocationally, very infinite! Suppose $\Pi$ is countably infinite, then (for simplicity) starting in the empty \state $(\emptyset, \emptyset) \in \carriers \views$. There are now exactly $|\Pi|$ possible successors, one for each argument. More generally, if an infinite number of arguments have already been introduced into the \state, and in at least some view $V_a$, there are arguments with infinite branching, we might have an infinite number of arguments to introduce, any number of which can be introduced in possibly an infinite number of ways (as per ref. N).

\begin{definition}\label{def:depth} The \emph{white modal depth} of $\phi \in \dlangm$ is $\depth{\phi} \in \mathbb N$, which is defined inductively as follows 
\begin{align*}
\depth{\alpha} \quad & := \quad 0 & \text{no white connectives in these formulas}\\
\depth{\cdia \alpha} \quad & := \quad 0 \\
\depth{\neg \phi} \quad &:= \quad \depth{\phi} & \text{depth is deepest nesting of }\\
\depth{\phi \wedge \psi} \quad &:= \quad \max\{\depth{\phi}, \depth{\psi}\} & \text{white connectives}\\
\depth{\adia \phi} \quad &:= \quad 1 + \depth{\phi} \\
\depth{\ddia p \phi} \quad &:= \quad 1 + \depth{\phi} \\
%\depth{\ddia \pi \phi} \quad &:= \quad \depth{\tau(\ddia \pi \phi)} & \text{unwrap complex programs}\\
\end{align*}
\end{definition}

We let $\restr \Pi \phi$ denote the set of arguments occurring in $\phi$ in sub-formulas from $\lblack$. Notice that given a state $q \in \carriers \views$, the satisfaction of a formula of the form  $\phi = \cdia \alpha$ at the AF encoded by $q$ is not dependent on the entire digraph $q = (q_S,q_E)$.

Indeed, this is what motivated our definition of normality for an argumentation semantics, leading to the following simple lemma, which is the first step towards shrinking Kripke structures for the purpose of model checking. Given a model $\kmod \views \sem$ and a state $q \in \carriers \views$, we let $\comp q \Phi$ denote the digraph consisting of all connected components from $q$ which contains a symbol from $\Phi$. Then we obtain the following.

\begin{lemma}\label{lemma:comp}Given a semantics $\sem$ and two bases $\views$ and $\views'$, we have, for any two states $q \in \kmod \views \sem$ and $q' \in \kmod {\views'} \sem$ and for any formula $\phi \in \dlangm$ with $\depth \phi = 0$:
$$
\big(\comp q {\restr \Pi \phi} = \comp {q'} {\restr \Pi \phi}\big) \Rightarrow \big(\kmod \views \sem,q \vDash \phi \Leftrightarrow \kmod {\views'} \sem, q'  \vDash \phi \big)$$
\end{lemma}

In order to complete our argument in this section, we will make use of $n$-bisimulations modulo a set of symbols.

\begin{definition}\label{def:bisim} Given two models (with possibly different bases, but with common set of symbols $\Pi$ and semantic $\sem$) $K_\views = \langle \carriers \views, R, \sem \rangle$ and $K_\views' = \langle \carriers {\views'}, R', \sem\rangle$, states $q \in \carrier$ and $q' \in \carrier'$, a natural number $n$ and a set $\Phi \subseteq \Pi$, then we say that $q$ and $q'$ are $n$-bisimilar modulo $\Phi$ (denoted $(K_\views, q) ~ \bisim_n^\Phi~ (K_{\views'}, q')$), if, and only if, there are $n+1$ relations relation $Z_n \subseteq Z_{n-1} \subseteq \dots \subseteq Z_0 \subseteq \carrier \times \carrier'$ such that
\begin{enumerate}
\item $q Z_n q'$, 
\item whenever $(v, v') \in Z_0$, then $C(v, \Phi) = C(v', \Phi)$, 
\item whenever $(v, v') \in Z_{i+1}$ and $vRu$, then there is a $u'$ s.t. $v'R'u'$ and $uZ_{i}u'$, 
\item whenever $(v, v') \in Z_{i+1}$ and $v'R'u'$, then there is a $u$ s.t. $vRu$ and $uZ_i u'$.
\end{enumerate}
\end{definition}

Let us now also define a particular subset of arguments, the arguments which have at most distance $n$ from some given set of arguments: 

\begin{definition}\label{def:vicinity}Given a \basis $\views = \viewsv$, a subset $\Phi \subseteq \Pi$ and a number $n$, the $n$-vicinity of $\Phi$ is $D(\views, \Phi, i) \subseteq \Pi$, defined inductively as follows
\begin{align*}
D(\views, \Phi, 0) &= \Phi \\
D(\views, \Phi, n+1) &= D(\views, \Phi, n) \\ 
&\cup \left\{~p \in \Pi ~|~ \exists q \in D(\views, \Phi, n): \{(p,q),(q,p)\} \cap \bigcup_{a \in \agents}V_a \not = \emptyset ~\right\}\\
\end{align*}
\end{definition}

Notice that as long as $\Phi$ is finite and all agents' views have finite branching, then the set $D$ is also finite. Also notice that an equivalent characterization of the set $D(\viewsv,\Phi,i)$ can be given in terms of paths as follows: an argument $p \in \Pi$ is in $D(\views,\Phi,i)$ if, and only if, there is a path $p=x_1x_2\ldots x_n$ in $\bigcup_{a \in \agents}V_a$ such that $x_n \in \Phi$ and $n \leq i$ (we consider an argument $p$ equivalently as an empty path at $p$). 

\begin{definition}\label{def:shrink} Given a formula $\phi \in \dlangm$. Let $\viewsv$ be a possibly infinite \basis, we define $\shrink \phi \viewsv$ such that
\begin{itemize}
\item for every $a \in \agents$, $\shrink \phi {V_a} ~:=~ V_a \cap D(V_a, \restr \Pi \phi, \depth{\phi})$
\end{itemize}
\end{definition}

Notice that the Kripke model for $\rho(\views)$ will have finite branching as long as the argument symbols in the $\depth{\phi}$-vicinity of the argument symbols in $\phi$ have finite branching in all agents' views. In the following, we
 will show that for any finitely branching $\views$ and normal $\sem$, we have $\kmod \views \sem, (\emptyset,\emptyset) \models \phi$ if, and only if, $\kmod {\shrink \phi \views} \sem, (\emptyset,\emptyset) \models \phi$. 

\begin{theorem} Let $\views$ be an arbitrary \basis, and and $\phi \in \dlangm$. 
$$ \left(\kmod \views \sem, (\emptyset,\emptyset)\right) \quad   \bisim_{\depth{\phi}}^{\restr \Pi \phi}  \quad \left(\kmod {\shrink \phi \views} \sem, (\emptyset,\emptyset)\right)$$
\end{theorem}

%\begin{proof} 
%\paragraph{Outline:} First we define the relations $Z_i$ which will witness the $n$-bisimulation. Towards this, we view both models as trees, i.e., that each state has a unique successor. The tree can be seen as a set of paths. We are interested in the finite postfixes of length up to $n$. The nodes in the path can be seen, rather than as graphs (AFs), as sets of colors of argument symbols. I.e., a node is now identified with a sequence of sets of vectors, where the first entry, representing the root, is the singleton set consisting of the $\Pi$-vector containing only $\hf$-s. 
%\end{proof}

\begin{proof}
Let $\kmod \views \sem$ be an arbitrary model and let $T$ denote its tree representation, while $T'$ denotes the tree representation of $\kmod {\rho_\phi(\views)} \sem$.

We take $n = \depth{\phi}$ and let $\Phi$ be the atoms occurring in $\phi$ inside the scope of some $\cdia$-operator. Moreover, for brevity, we denote $D = D(\views, \Phi, n)$. % In the proof we'll refer to the set of possible attacks among arguments in $D$ as $D^2$.

\paragraph{Definition of $(Z_i)_{(0 \leq i \leq \depth{\phi})}$: } 
We define all the relations $Z_i$ inductively using the tree-representations as follows.
\begin{description}
\item[Base case: ] ($i = 0$) For all $0 \leq i \leq n$, we let $\epsilon Z_i \epsilon$. 
\item[Induction step: ] ($0 < i \leq n$) For all $y = x;(v, X) \in T$ and $y' = x';(v', X') \in T'$, both of length $i$, with $x (Z_{i+1}) x'$. We let, for every $k \leq i$, $y (Z_k) y'$ if, and only if, $v = v'$, \emph{and} $X \cap (D \times D)= X'$.
\end{description}

Notice that if $x (Z_i) x'$, then $S(x) = S(x')$ and $|S(x)| \leq (n - i)$. Moreover, by consulting Definition \ref{def:treerep} it is not hard to see that for all $q \in  \carriers \views , q' \in \carriers {\rho(\views)}$ we have, for all $0 \leq i \leq n$ and all $q \in \carriers \views,q' \in \carriers {\rho(\views)}$: 
$$
\forall x_1,x_2 \in \gamma(q): \forall x'_1,x'_2 \in \gamma(q'): x_1(Z_i)x_2 \iff x'_1 (Z_i) x'_2 
$$
This means, in particular, that the following lifting of $(Z_i)_{0 \leq i \leq n}$ to models is well-defined, for all $q \in \carriers \views, q' \in \carriers {\rho(\views)}$ and all $0 \leq i \leq n$:
$$
q(Z_i)q' \iff x(Z_i)x' 
$$
for some $x \in \gamma(x), x' \in \gamma(q')$.

Next we show that $(Z_i)_{0 \leq i \leq n}$ so defined is an n-bisimulation between $\kmod \views \sem$ and $\kmod {\rho(\views)} \sem$.

\paragraph{$(Z_i)_{0 \leq i \leq n}$ witnesses $n$-bisimulation:} We address all the points of the definition of $n$-bisimulation modulo $\Phi$ in order.
\begin{enumerate}
\item Clearly, $(\emptyset, \emptyset) Z_n (\emptyset, \emptyset)$. Hence the first condition of the definition is satisfied. 
\item Consider any arbitrary states $q,q'$ and let $x = x_1;x_2;\dots;x_m$ and $x' = x_1';x_2';\dots;x_m'$ be the corresponding nodes from $T,T'$ that witnesses to $q(Z_0)q'$. By definition of $Z_0$ we have $S(x) = S(x')$, but it is possible that we have $E(x) \not = E(x')$. However, we must have $C(\af(x), \Phi) = C(\af(x'), \Phi)$, and to see this, it is enough to observe that as $m \leq n$, each of $x$ and $x'$ contains at most $n$ nodes. Then, since $\af(x) = q$ and $\af(x') = q'$ are the same on $D$, and the distance from $\Pi \setminus D$ to $\Phi$ is greater than $n$. That is, any path from an argument in $\Pi \setminus D$ to an argument in $\Phi = \restr \Pi \phi$ would be a path consisting of at least $n+1$ nodes. It follows that no element from $\Phi$ can be in a connected components containing elements outside of $D$.
\item Consider now $q,q'$ corresponding to $x$ and $x'$ such that $x (Z_{i + 1}) x'$. Notice that $(q,r) \in \rels \views(\exists)$ if, and only if, there is a $(p, X)$ such that $x R (x;(p, X))$. So all we need to show is that $X \cap (D \times D)$ is in $\update {\rho_\phi(\views)} {x'} p$. Then it will follow that there is a successor to $x'$, namely $(p, X \cap (D \times D))$, with $(x')R'(x';(p, X \cap (D \times D)))$. This is a straightforward consequence of the Definition \ref{def:shrink} of $\rho$. The argument for the particular sub relations $\rels \views(p)$ is analogous. 
\item Finally consider $q,q'$ corresponding to $x$ and $x'$ such that $x (Z_{i + 1}) x'$ for $(p, X)$ such that $x' R (x';(p, X'))$. Again we need to ensure that there is an $X \in \update \views x p$ such that $X' = X \cap (D \times D)$, and again this follows from the Definition \ref{def:shrink} of $\rho$. The argument for the particular sub relations $\rels \views(p)$ is analogous. 
\end{enumerate}
\end{proof}

\begin{proposition} Let $\phi \in \dlangm$ and $\views, \views'$ arbitrary bases. If states $q \in \kmod \views \sem$ and $q' \in \kmod {\views'} \sem$ are $\depth{\phi}$-bisimilar modulo $\restr \Pi \phi$, then $\kmod \views \sem, q \models \phi ~\Leftrightarrow~ \kmod {\views'} \sem , q' \models \phi$. Or, succinctly
$$ \left((\kmod \views \sem, q) ~\bisim_{\depth{\phi}}^{\restr \Pi \phi}~ (\kmod {\views'} \sem, q')\right) ~\Rightarrow~ \left(\kmod \views \sem, q \models \phi ~\Leftrightarrow~ \kmod {\views'} \sem, q' \models \phi\right).$$
\end{proposition}

\begin{proof} The proof is by induction on $\depth{\phi}$. 
\begin{description}
\item[Base case:] ($\depth{\phi} = 0$) There are no white connectives, and our states, $q$ and $q'$, are clearly 0-bisimilar modulo $\Phi$. It is also easy to see, consulting Definition \ref{def:main}, that the truth of a formula of modal depth $0$ is only dependent on the AF $q$. Then it follows from the fact that $\sem$ is assumed to be normal that the truth of $\phi$ is in fact only dependent on $C(q, \Phi)$. From $q (Z_0) q'$, we obtain $C(q, \Phi) = C(q', \Phi)$ and the claim follows. 
\item[Induction step:] ($\depth{\phi} > 0$) We skip the boolean cases as these are trivial, so let $\phi := \adia \psi$ (the case of white connectives with an explicit argument is similar). Suppose $\depth{\phi} = i + 1$ and $q (Z_{i+1}) q'$. Suppose further that $\kmod \views \sem, q \models \adia \psi$. Then there is a successor of $q$, $v \in succ(q)$ such that $\kmod \views \sem, v \models \psi$. All successors of $q$ will be $i$-bisimilar to a successor of $q'$ (point 3. of Definition \ref{def:bisim}). So we have $(\kmod \views \sem, v)~\bisim_i^{\Phi}~(\kmod {\views'} \sem, v')$. As $\depth{\psi} < \depth{\adia \psi}$ we can apply our induction hypothesis to obtain $\kmod {\views'} \sem, v' \models \psi$, and $\kmod {\views} \sem, q' \models \adia \psi$ as desired.
\end{description}

\end{proof}

\section{Conclusion and future work}\label{sec:concfut}

We have argued for a logical analysis of deliberative processes by way of modal logic, where we avoid making restrictions that may not be generally applicable, and instead focus on logical analysis of the space of possible outcomes. The deliberative dynamic logic (\acro{ddl}) was put forth as a concrete proposal, and we showed some results on model checking.

We notice that \acro{ddl} only allows us to study deliberative processes where every step in the process is explicitly mentioned in the formula. That is, while we quantify over the arguments involved and the way in which updates take place, we do not quantify over the \emph{depth} of the update. For instance, a formula like $\Diamond \Box p$ reads that there is a deliberative update such that no matter what update we perform next, we get $\phi$. A natural next step is to consider instead a formula $\Diamond \Box^\ast \phi$, with the intended reading that there is an update which not only makes $\phi$ true, but ensures that it remains true for all possible future \emph{sequences} of updates. Introducing such formulas to the logic, allowing the deliberative modalities to be iterated, is an important challenge for future work.  Moreover, we would also like to consider even more complex temporal operators, such as those of computational tree logic, or even $\mu$-calculus.

Finding finite representations for the deliberative truths that can be expressed in such languages appears to be much more challenging, but we would like to explore the possibility of doing so.

Also, we would like to explore the question of validity for the resulting logics, and the possibility of obtaining some compactness results. Indeed, it seems that if we introduce temporal operators we will be able to express truths on arbitrary points $q \in \carriers \views$ by corresponding formulas that are true at $(\emptyset,\emptyset)$, thus capturing the way in which complete assent can be faithfully captured by a finite (albeit unbounded) notion of iterated deliberation. 

If the history of the human race is anything to go by, it seems clear that we never run out of arguments or controversy. But it might also be that some patterns or structures are decisive enough that they warrant us to conclude that the \emph{truth} has been settled, even if deliberation may go on indefinitely. A further logical inquiry into this and related questions will be investigated in future work.

\bibliographystyle{abbrv}
\bibliography{./white-japan/cites,more_cites}
\end{document}


%% Big (original) graph
%% \newpage
%% \begin{figure}[ht]
%% \centering
%% \input{img/big-graph.pdf_tex}
%% \caption{Inkscape to \LaTeX -test}
%% \end{figure}
